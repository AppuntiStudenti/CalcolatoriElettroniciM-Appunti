\documentclass[a4paper]{book}
\usepackage[italian]{babel}
\usepackage[utf8]{inputenc}
\usepackage[T1]{fontenc}
\usepackage[sc]{mathpazo} % Palatino!
\linespread{1.05}         % Palatino needs more leading (space between lines)
\usepackage{graphicx}
\usepackage{amsmath}
\usepackage{indentfirst}
\usepackage{amsthm}
\usepackage{amssymb}
\usepackage{booktabs}
\usepackage{listings}
\usepackage{wrapfig}
\usepackage{bbding}
\usepackage{float}
\usepackage[a4paper,top=2.5cm,bottom=2.5cm,left=2cm,right=2cm,bindingoffset=5mm]{geometry}
\usepackage[font=small,format=hang,labelfont={sf,bf}]{caption} % Per un diverso carattere per le didascalie


\newenvironment{esercizio}{\subsection{Esercizio}\sffamily}{}

\author{Ingegnere Pazzo \\ http://ingegnerepazzo.wordpress.com/}
\title{Appunti di Sistemi d'elaborazione dell'informazione \\ (Prima versione)}

\begin{document}

\maketitle
\thanks{Si consiglia di affiancare il materiale presente in questo riassunto agli appunti presi a lezione. Questo perché (ovviamente!) non ho alcuna presunzione di esaustività, né di assoluta correttezza: nonostante una prima revisione, potrebbero infatti essere ancora presenti molti errori e imprecisioni. Si ringrazia il prof. Tullio Salmon Cinotti per avermi permesso di usare, in questi appunti, alcune immagini tratte dalle sue \textit{slides}.}
\tableofcontents
\chapter{Introduzione}
\label{cha:Introduzione}

\section{Cosa determina le prestazioni di un calcolatore}
\label{sec:cosaDeterminaPrestazioni}

Un'architettura di elaborazione\footnote{Faremo costantemente riferimento al modello basato sulla macchina di Von Neumann.} è costituita da tre elementi fondamentali:
\begin{itemize}
\item l'\textbf{ISA} (\textit{Instruction Stack Architecture}), ovvero il set di istruzioni computabile dalla macchina;
\item la \textbf{struttura} (v. esempio in figura \ref{fig:strutturaSchematica}), cioè l'insieme di blocchi interconnessi fra loro che realizza il principio di funzionamento del sistema;
\begin{figure}[!h]
\centering
\includegraphics[width=0.8\columnwidth]{img/strutturaSchematica}
\caption{Una schematizzazione essenziale della struttura di un calcolatore}
\label{fig:strutturaSchematica}
\end{figure}
\item la \textbf{realizzazione circuitale} del sistema, cioè com'è fatto "'elettronicamente'".
\end{itemize}

Questi tre elementi impattano significativamente sulle prestazioni del nostro elaboratore, quantificabili attraverso il parametro $CPU_{time}$, cioè il tempo che impiega la macchina ad eseguire un certo \textit{task}:
\[
CPU_{time} = N_{istruzioni} \cdot CPI_{medio} \cdot T_{clock}
\]
Il numero di istruzioni ($N_{istruzioni}$) dipende dell'ISA, il $CPI_{medio}$ dalla struttura hardware del nostro sistema, il $T_{clock}$ (periodo di clock) dalla tecnologia di realizzazione (cioè dall'elettronica sottostante la fabbricazione dell'hardware).

Tecnologia e struttura interna evolvono nell'ottica di aumentare le prestazioni e ridurre il consumo/operazione
elementare. In particolare, si cerca di trovare il miglior compromesso fra potenza e consumo tramite i seguenti accorgimenti:
\begin{itemize}
\item riduzione delle tensioni di alimentazione;
\item definizione di diversi stati di funzionamento in modo da
alimentare selettivamente a divisione di tempo solo i blocchi
istante per istante necessari;
\item variazione della frequenza di funzionamento in funzione del
carico computazionale;
\item variazione della tensione di alimentazione in funzione della
frequenza istantanea di funzionamento;
\item \textit{Power management} esteso all'intero sistema, non solo alla
CPU.
\end{itemize}


\subsection{L'ISA}
\label{sec:isa}

Il linguaggio macchina (L.M.) (detto anche \textit{instruction set architecture} - ISA) è il livello dell'architettura della CPU visibile a chi sviluppa i compilatori e a chi programma in \textit{assembler}. L'ISA è costituita dall'insieme delle istruzioni eseguibili e dal loro formato binario. Con le istruzioni del linguaggio macchina si accede alle risorse interne del calcolatore (memoria, registri, tabelle e descrittori, variabili di stato, \textit{flag}, etc\ldots). Tra le risorse interne al calcolatore, solo quelle accessibili attraverso l'ISA possono essere rese visibili e controllabili dal software.

Per comodità in generale non rappresenteremo le istruzioni macchina con zeri e uni (cioè come vengono viste dal calcolatore), né con cifre esadecimali in forma simbolica; il linguaggio che useremo per rappresentare simbolicamente le istruzioni del linguaggio macchina è l'\textit{assembler}.

Le istruzioni prese in pasto da un calcolatore\footnote{In questo paragrafo faremo riferimento all'Intel 8086/8088 oppure al DLX.} hanno una struttura del tipo:
\begin{verbatim}
ADD   ALFA[DI+BX], AL
\end{verbatim}
In questo caso DI+BX è l'indice di ALFA, che è un vettore: l'operazione effettuata consiste nel sommare alla quantità presente in ALFA[DI+BX] (operando in memoria) la quantità AL. Con ALFA[DI+BX] indichiamo simbolicamente una posizione in memoria; la memoria, nell'8088, è di 1 MB ($20^{20}$ bit) ed è "'puntata'" da DS (\textit{data segment}, vedi fig. \ref{fig:DSmemoria}), vettore di 20 bit con i 4 bit meno significativi posti a 0: questo significa che è possibile indirizzare le celle di memoria, ampie $2^4 = 16$ byte ciascuna e dette \textit{paragrafi}, con soli 16 bit (4 cifre esadecimali) dei 20 totali.
\begin{figure}[!h]
\centering
\includegraphics[width=0.7\columnwidth]{img/DSmemoria}
\caption{Struttura della memoria e suddivisione in paragrafi}
\label{fig:DSmemoria}
\end{figure}

Un'altra possibile suddivisione logica della memoria è quella in pagine, ovvero in $2^8$ blocchi di $2^{12}$ byte (vedi fig. \ref{fig:pagineParagrafi}).

\begin{figure}[!h]
\centering
\includegraphics[width=0.4\columnwidth]{img/pagineParagrafi}
\caption{\textit{Base} e \textit{offset} per pagine e paragrafi}
\label{fig:pagineParagrafi}
\end{figure}

Si noti che la divisione in paragrafi e in pagine implica:
\begin{itemize}
\item un metodo di indirizzamento \textit{base} (indirizzo iniziale della cella di memoria) + \textit{offset} (indirizzo all'interno della cella);
\item che è possibile far sì che l'indirizzo di un blocco sia sempre un multiplo della dimensione del blocco stesso (\textit{indirizzi allineati}); questo comporta che, se prendiamo in considerazione il DLX (4 GigaByte di memoria da indirizzare, indirizzi di 32 bit), non necessiteremo di un sommatore a 32 bit per il calcolo dell'indirizzo "'effettivo'" (\textit{base} + \textit{offset}), potendo furbescamente sfruttare il concatenamento di \textit{blockID} + \textit{offset}. \\
Esempio (DLX, $n=32$, indirizzi allineati): devo accedere a una cella di memoria individuata dal \textit{blockID} che, per definizione, ha i $k$ bit meno significativi tutti a zero, e ho bisogno dell'informazione, che chiameremo PIPPO, presente ad un certo \textit{offset} lungo $k$ bit. Per trovare l'indirizzo di PIPPO basta una semplice operazione di concatenamento:
\begin{verbatim}
ind(PIPPO) [n bit] = Block_ID [n-k bit] ## Offset [k bit]
\end{verbatim}
Se invece i blocchi non sono allineati (es. 8088, dove sia la base che l'offset sono a 16 bit, con un DS di 20 bit), oppure se l'offset è maggiore della dimensione del blocco, il sommatore è imprescindibile.
\end{itemize}

Nell'8088 l'offset è calcolato dinamicamente ed è la somma di BX + DI + ALFA (vedi fig. \ref{fig:effectiveAddress}).

\begin{figure}[!h]
\centering
\includegraphics[width=0.5\columnwidth]{img/effectiveAddress}
\caption{Calcolo dell'\textit{effective address}}
\label{fig:effectiveAddress}
\end{figure}

\subsection{Struttura (architettura)}
\label{sec:strutturaArchitettura}

\begin{figure}[!h]
\centering
\includegraphics[width=0.7\columnwidth]{img/tipiArchitetture}
\caption{Schema riassuntivo della nomenclatura per tipologie d'architettura}
\label{fig:tipiArchitettura}
\end{figure}

Una macchina che fa uso di due operandi espliciti e di uno in memoria viene detta avente architettura 2-1. 
Esempio di istruzione di macchina 2-1 (detta anche M-R, memoria-registro): 
\begin{verbatim}
ADD   ALFA[DI+BX], AL
\end{verbatim}
Con una singola riga di codice effettuiamo una relativamente numerosa serie di operazioni elementari (memorizzazione, \textit{fetch}, somme, etc\ldots), le quali richiedono molti clock per essere eseguite; dunque:
\[
CPI_{medio}~~ \text{(clock per instruction: ALTO)} ~~~~~~~~~ N_{istruzioni}~~ \text{(numero di istruzioni: BASSO)}
\]
Si parla quindi di architettura CISC (\textit{Complex Instruction Set Computer}). \\

Il DLX ha invece un'architettura 3-0 (detta anche R-R, registro-registro), con istruzioni aventi tre operandi espliciti e nessuno in memoria:
\begin{verbatim}
ADD R5, R6, R7
\end{verbatim}
Per prelevare e depositare in memoria ci si appoggia quindi alle istruzioni di LOAD e STORE. \\

L'architettura influisce tantissimo sia sulle prestazioni che sul generale funzionamento del calcolatore: infine, con una stessa ISA possono sussistere strutture diversissime (ad esempio può essere più o meno realizzabile la \textit{pipeline} piuttosto che la struttura ad esecuzione sequenziale, etc\ldots).

\section{Tipi di istruzioni}
\label{sec:tipiIstruzioni}

In questo paragrafo considereremo l'architettura del DLX ($2^{32}$ byte di indirizzamento).

\begin{figure}[!h]
\centering
\includegraphics[width=0.65\columnwidth]{img/TipiIstruzioni}
\caption{Tipologie d'istruzione}
\label{fig:TipiIstruzioni}
\end{figure}
\begin{figure}[!h]
\centering
\includegraphics[width=\columnwidth]{img/listaistruzioni}
\caption{Schema di massima dei principali tipi di istruzioni}
\label{fig:listaistruzioni}
\end{figure}

\subsection{Di tipo R}
\label{sec:tipoR}

Si tratta delle tipiche istruzioni ALU, in cui sono coinvolti tre registri. Esempio:
\begin{verbatim}
ADD   R1, R2, R3
\end{verbatim}

\subsection{Di tipo I}
\label{sec:tipoI}

Sono le istruzioni con operando immediato (cui sono riservati 16 bit in complemento a 2, per un range - quindi - che va da -32K a 32K-1), come
\begin{itemize}
\item \textit{Load}/\textit{Store}. Ad esempio:
\begin{verbatim}
LB    R5, 20(R6)
\end{verbatim}
Questa istruzione\footnote{Equivalente a 
\[
R5_{[7\ldots 0]} \Leftarrow M[R6+20]
\]} mette in R5 il contenuto della cella di memoria il cui indirizzo è il contenuto del registro R6 incrementato di 20.
Oppure:
\begin{verbatim}
LDRS   ALFA(R6)
\end{verbatim}
ALFA è detta \textit{parte fissa}, mentre il contenuto di R6 è la \textit{parte variabile}: quel che si fa è accedere all'R6-simo elemento del vettore ALFA ed effettuarne il \textit{load}.
\item \textit{Branch condizionate}, ovvero salto \textbf{condizionato}. Esempio\footnote{Si tratta di una \textit{Branch Not Equal Zero}: in pratica, se in PIPPO vi è il valore 3, dobbiamo andare a PC+3 (per questo si dice che è un'istruzione \textit{PC-relative}).}
\begin{verbatim}
BNEZ   R5, PIPPO
\end{verbatim}
Altri esempi (\textit{Branch Not Equal Zero})\footnote{Se R10 è diverso da 0 saltiamo a PIPPO.}:
\begin{verbatim}
BNEQZ   R10, PIPPO
\end{verbatim}
Che è equivalente a (\textit{Jump Equal})\footnote{Se R5=R6 saltiamo a PIPPO.}
\begin{verbatim}
JEQ    R5, R6, PIPPO
\end{verbatim}
se R10 è stato ricavato con questa istruzione (\textit{Set Equal Zero})\footnote{Mettiamo in R10 il risultato del confronto fra gli operandi R5 e R6.}:
\begin{verbatim}
SEQ    R10, R5, R6
\end{verbatim}

\item \textit{Jump Register} (JR);
\item \textit{Jump and Link register} (JALR);
\item ALU con operando immediato, ad esempio:
\begin{verbatim}
ADDI   R1, R2, 3
\end{verbatim}
Per questo tipo di operazioni viene coinvolta la ALU, che prende in pasto due operandi da 32 bit: siccome l'operando immediato è una quantità a 16 bit, i rimanenti bit sono rappresentati dal bit del segno ripetuto 16 volte.
\end{itemize}

\subsection{Di tipo J}
\label{sec:tipoJ}

Si tratta di istruzioni di \textit{jump} (salto \textbf{incondizionato}), ancora una volta \textit{PC-relative} con l'operando immediato presente nella finestra\footnote{Ampia 64 MB e centrata sul PC.}:
\[
-2^{25} \Rightarrow 2^{25}-1
\]
Quando è necessario effettuare un salto (\textit{jump}) l'indirizzo di ritorno viene salvato nel registro R31 ma, se per qualche motivo nel frattempo venisse eseguita un'altra procedura, l'indirizzo di ritorno in R31 verrebbe sovrascritto e noi saremmo fregati. Quanto detto si riflette nell'assenza del cosiddetto \textit{nesting}, cosicché deve curarsene il programmatore (facendo uso, ad esempio, di altri registri come R30). Se teniamo conto di questo aspetto, come ci regoliamo con gli \textit{interrupt}? Gli \textit{interrupt} sono chiamate a procedure "'implicite'", scatenate da un evento o un dispositivo esterno (come ad esempio una periferica): l'indirizzo di ritorno, in questo caso, non viene salvato in R31 bensì in IAR(\textit{Interrupt Address Register}). 
IAR e R31 sono quindi i due registri deputati al salvataggio degli indirizzi di ritorno.

Come facciamo per regolare il ritorno dalle eccezioni (ad es. divisione per zero, al sopraggiungere della quale la CPU si arrabbia e si sfoga sull'utente generando - appunto - un'eccezione)? Esiste un'istruzione ad-hoc
\begin{verbatim}
RFE   (Return From Exception)
\end{verbatim}
all'esecuzione della quale si salva l'indirizzo corrente in IAR. 

\input{Pipeline.tex}
\chapter{Intel Architecture 32 bit, alcuni aspetti}
\label{cha:ia32}

\section{Bus e prime problematiche con la gerarchia delle memorie}
\label{sec:busMemorie}

Osserviamo la figura \ref{fig:strutturaInternaBus}; notiamo la presenza di due BUS principali, uno per la memoria (bus dati) e uno per le periferiche (IOB), separati da un \textit{bridge} fungente da ponte: si noti che essi hanno dimensione diversa (uno è a 64 bit, l'altro è a 8 bit). Ogni bus trasferisce una quantità di informazioni pari alla propria capacità ad ogni ciclo di clock, tuttavia non è detto che un singolo colpo di clock sia sufficiente per trasportare un intero dato. Prendiamo in considerazione le cosiddette "'linee di cache'"\footnote{Come nel caso delle memorie a livelli gerarchici superiori, anche la cache è suddivisa in blocchi di uguale dimensione. Nel caso delle cache questi blocchi sono detti \textit{linee di cache}. Nel Pentium e nel P6 (bus dati da 64 bit) la dimensione di ogni linea di cache è di 32 byte ed è identificabile dai 27 bit BA[31..5] (base) dell'indirizzo fisico costituito da 32 bit (27 bit base + 5 bit offset).}, unità grandi 32 byte in cui quest'ultima memoria è suddivisa: dovendo essere trasportate dal bus dati, si necessiterà di quattro cicli di bus per completare un singolo trasferimento. 

\begin{figure}[!h]
\centering
\includegraphics[width=0.85\columnwidth]{img/strutturaInternaBus}
\caption{Struttura interna della CPU}
\label{fig:strutturaInternaBus}
\end{figure}

Ogni ciclo di bus (dati), infatti, trasporta solo 8 dei 32 byte: da qui la necessità di effettuare cicli di bus multipli e di un segnale \textit{ready} (vedi fig. \ref{fig:segnaleReady}) in grado di indicare quando un dato è pronto e completamente trasferito. Inoltre, il meccanismo di generazione del segnale di \textit{ready} deve farci rispettare correttamente le tempistiche d'accesso alla memoria: se disponiamo, ad esempio, di una memoria con tempo d'accesso 100 ns e di un processore di 1 GHz dovremo aspettare 100 clock prima di poter effettuare le nostre operazioni.

\begin{figure}[!h]
\centering
\includegraphics[width=0.35\columnwidth]{img/segnaleReady}
\caption{Funzionamento del segnale di \textit{ready}}
\label{fig:segnaleReady}
\end{figure}

Siamo quindi già arrivati a distinguere fra due diverse tipologie di cicli di bus:
\begin{itemize}
\item ciclo singolo: trasferiamo un unico dato;
\item ciclo \textit{burst}: trasferiamo quattro dati invece che uno solo (anche statisticamente si è visto che è la scelta migliore).
\end{itemize}
Purtroppo andare a leggere una periferica è un'operazione lentissima e abbiamo tempi d'accesso esorbitanti (confrontati con quelli d'accesso alla \textit{cache}); se il bus è bloccante ne segue che l'intera macchina lo è: volendo tuttavia poter usufruire dell'algoritmo di Tomasulo, il quale ci permette di eseguire fuori ordine, dobbiamo permettere che la memoria e l'I/O possano rispondere in ritardo (\textit{split transaction cycle}).

Inoltre, la CPU deve sapere a che ciclo di bus si riferisce un determinato dato: questo ci spinge ad inserire segnali di controllo e bit aggiuntivi. La cosa diventa ancora più complicata se abbiamo a che fare con sistemi \textit{multiprocessor}: dobbiamo infatti prevedere l'arbitraggio del bus, che è unico ed organizzato a mo' di \textit{pipeline} (cosicché serve dell'\textit{hardware} particolare per ogni stadio della pipeline e ulteriori segnali aggiuntivi... Non ne usciamo più!); infine possiamo avere in \textit{cache} delle copie di dati non consistenti (\textit{stail data}, dato presente in due memorie con valore diverso). Un analogo inconveniente può presentarsi per diversità dei dati fra la RAM e la \textit{cache} (problema di \textit{cache coherency}).
I componenti che si occupano di queste problematiche ed inoltre gestiscono le risorse, spegnendo la CPU in caso di emergenza o surriscaldamento, aumentando o riducendo la frequenza di \textit{clock}, etc\ldots sono i \textit{master} del bus (vedi fig. \ref{fig:busDintorni}.).

\begin{figure}[!h]
\centering
\includegraphics[width=0.75\columnwidth]{img/busDintorni}
\caption{Il bus e gli agenti interagenti con esso}
\label{fig:busDintorni}
\end{figure}

Morale della favola: un sistema complesso deve gestire moltissime cose e deve essere in grado di rilevare le occorrenze proibite, evitando che vengano eseguite e segnalando il \textit{software} con opportuni \textit{interrupt}\footnote{Prevenire è meglio che curare (cit.): serve obbligatoriamente un'opportuna \textit{routine} di gestione degli interrupt.}.

\section{IA32, protezione e gerarchia delle memorie}
\label{sec:gerarchia_memorie}

Osserviamo più da vicino al struttura del nostro processore Intel (vedi fig. \ref{fig:ia32}).

\begin{figure}[!h]
\centering
\includegraphics[width=\columnwidth]{img/IA32}
\caption{Ambiente di esecuzione di una applicazione
nell'architettura Intel a 32 bit.}
\label{fig:ia32}
\end{figure}

Qualche nota sui registri:
\begin{itemize}
\item ESP: \textit{Stack Pointer}. Si tratta dell'\textit{offset} nel registro di \textit{stack} dove siamo posizionati in un determinato momento;
\item EDI/ESI: registri indice per i vettori (max. 4 GB);
\item EIP: è l'\textit{Extended Instruction Pointer}, del quale presto scopriremo l'utilità;
\item sei registri di segmento entro i quali si manifesta la protezione\footnote{Una CPU protetta è una CPU in grado di separare i \textit{task} in base agli utenti che li hanno avviati.} della CPU (vedi capitolo \ref{cha:protezione}): essi indicano la tipologia, le proprietà, la dimensione e chi abbia il diritto d'accesso ad un certo segmento. Le CPU IA32 dispongono di sei registri di segmento: CS (\textit{code segment}: contiene il descrittore del blocco di codice in esecuzione), DS, ES, FS, GS ed SS (\textit{stack segment}: contiene il descrittore dello \textit{stack} in uso, v. paragrafo \ref{sec:controlliCPU}). La struttura di un registro di segmento è riportata in figura \ref{fig:regSegmento}.
\end{itemize}

Come nell'8086\footnote{Nell'8086 il \textit{Program Counter} era la coppia CS\#0000 + IP e puntava alla memoria di un 1 MB. Il PC era ottenuto da segmento + offset e si trovava in CS, con indirizzo iniziale FFFF0. In IA32, come si legge poco oltre, troviamo il descrittore di segmento, con indirizzo iniziale FFFFFFF0.}, per poter accedere a un segmento è necessario chi il relativo descrittore (per capire meglio cosa siano i descrittori si veda il paragrafo \ref{sec:descrittori}) sia stato preventivamente caricato in un registro di segmento. In particolare, CS contiene il segmento di codice del programma che in quel momento sta usando un'eventuale istruzione.

\begin{figure}[!h]
\centering
\includegraphics[width=0.84\columnwidth]{img/regSegmento}
\caption{Registro di segmento}
\label{fig:regSegmento}
\end{figure}

Nell'8086 ogni task ha 4 GB di memoria virtuale interamente dedicati a lui e sussistono sempre, anche se il quantitativo di RAM che ho nel sistema è molto inferiore: grazie a questo ingegnoso \textit{escamotage} siamo in grado di svincolare la dimensione del programma da quella della memoria fisica.

\begin{figure}[!h]
\centering
\includegraphics[width=\columnwidth]{img/slideSacra}
\caption{La gerarchia delle memorie}
\label{fig:slideSacra}
\end{figure}

A titolo informativo, mentre lo spazio dei segmenti e la \textit{cache} sono organizzati come matrici bidimensionali (a 2 coordinate), la memoria virtuale è lineare (1 coordinata).

Vediamo ora qualche esempio di codice in linguaggio macchina (NOTA: nei compiti segmento di codice, di dati e di \textit{stack} costituiscono un trinomio obbligatorio da definire!). \\
Segmento di \textbf{codice}:
\begin{verbatim}
             ER@10000H
Entry point: MOV DS, SEG_DATI
             ADD AL, DS:5 (aggiunge 5 ad AL)
             MOV DS: ALFA, AL 
             MOV EDI, 0 (mette a 0 EDI)
             MOV DS: BETA(EDI), EBX
             PUSH ...
             CALL ... (servirà supporto per il nesting!)
\end{verbatim}
Segmento \textbf{dati}:
\begin{verbatim}
         SEG_DATI SEGMENT RW@8000H (RW => leggere/scrivere)
8000H |  ALFA DB (DB => define byte)
8001H |  BETA DD 4DUP    (DD => double word = 4 byte)
...   |                  (4DUP => serve a duplicare 4 volte qualcosa)
      |                  (4 byte x 4 = 16 = 10H in esadecimale!)
8011H |  GAMMA DB
...
         ENDS
\end{verbatim}
Segmento di \textit{\textbf{stack}}:
\begin{verbatim}
STK_SEGMENT <..., ...>
\end{verbatim}

Notiamo che:
\begin{itemize}
\item la variabile ALFA è individuata dall'indirizzo logico <SEG\_DATI, 0>;
\item la variabile BETA(0) è individuata all'indirizzo logico <SEG\_DATI, 1>;
\item la variabile BETA(1) è individuata all'indirizzo logico <SEG\_DATI, 5>. Il 5 è stato calcolato ricordando che ogni elemento della struttura dati BETA è una \textit{double word} (4 byte);
\item gli indirizzi 8000H, 8001H, etc\ldots non sono indirizzi logici ma indirizzi nello spazio virtuale.
\end{itemize}

\begin{figure}[!h]
\centering
\includegraphics[width=0.8\columnwidth]{img/pipelineIndirizzi}
\caption{Pipeline e generazione degli indirizzi.}
\label{fig:pipelineIndirizzi}
\end{figure}

La pipeline come pensata in figura è in grado di funzionare su macchine M-R (come quelle Intel) e di effettuare la traduzione fra indirizzo logico e indirizzo virtuale. Ad esempio, ALFA ha indirizzo logico\footnote{Il programmatore (e i compilatori) vedono uno spazio di indirizzamento segmentato, il che significa che ogni istruzione e ogni dato vengono localizzati attraverso una coppia di coordinate: il segmento di appartenenza e la posizione (offset) all'interno del segmento (come in IA16); l'offset nel segmento può essere determinato dal compilatore (es. etichetta o variabile) oppure calcolato a \textit{runtime} (es. indirizzamento tramite registro base e registro indice).
} <SEG\_DATI, 0> ma indirizzo virtuale 8000H (+ offset pari a 0) calcolato facendo l'usuale operazione di somma base\footnote{DS concatenato con 4 zeri nell'8086; nell'IA32 è l'indirizzo virtuale calcolato nello stadio di AG, all'interno del quale è necessariamente utile avere i registri di segmento.} + offset.

\subsection{I \textit{Page Fault} e il reperimento dei blocchi}
\label{sec:pageFault}

Ogni volta che si esegue un \textit{task} è come se esistesse un file, all'interno dell'hard disk e grande 4 GB (ma magari "'scritto'" anche solo in pochi KB), che viene portato in memoria fisica (RAM): siccome la dimensione della RAM può essere anche molto inferiore a 4 GB, tale file viene smembrato in blocchi tutti uguali (pagine) di dimensione 4 KB. Se in memoria fisica viene ricercata una pagina che non c'è, allora si genera un \textit{page fault}, al quale si risponde caricando i dati cercati da memoria virtuale a memoria fisica.
La CPU, quando deve cercare qualcosa nella \textit{cache}, controlla i primi 27 bit dell'indirizzo fisico\footnote{Si ricorda che 27 dei 32 bit rappresentano la base e i 5 rimanenti l'offset.}.

\begin{figure}[!h]
\centering
\includegraphics[width=0.6\columnwidth]{img/hdrc}
\caption{Indirizzi e rispettive locazioni}
\label{fig:hdrc}
\end{figure}

La traduzione degli indirizzi (vedi fig. \ref{fig:hdrc}) avviene attraverso tabella associativa (passaggio indicato col numero 1, IF-IC) oppure attraverso un \textit{Translation Look-Aside Buffer} (passaggio 2, IV-IF).
L'assenza di una pagina in memoria può avvenire ad ogni livello:
\begin{itemize}
\item \textit{page fault} (tra IV e IF);
\item \textit{miss} (tra IF e IC);
\item \textit{segment non present fault} (tra IL e IV).
\end{itemize}
Ovviamente l'architettura che gestisce tutte queste tipologie di \textit{miss} dev'essere adeguatamente veloce e deve sussistere una collaborazione HW-SW per gestire:
\begin{enumerate}
\item i \textit{cicli burst} per portare i dati in \textit{cache};
\item l'allocazione delle pagine in memoria centrale, attività della quale si occupa in genere il sistema operativo;
\item il reperimento delle informazioni non ancora presenti in memoria virtuale (\textit{segment non present fault});
\item il controllo sugli accessi in memoria (protezione);
\item la definizione di processi e delle risorse di memoria ad essi associati;
\item la separazione tra processi (protezione);
\item la commutazione tra un processo e l'altro (\textit{task switching});
\item le tecniche d'accesso all'hard disk, che è risaputamente un collo di bottiglia per le prestazioni.
\end{enumerate}
Tutto ciò (soprattutto i punti 1, 2, 3 e 8) è causa una gran mole di \textit{overhead}, ma fortunatamente il numero di \textit{miss} non è così grande grazie al principio di località.

\section{Descrittori di segmento}
\label{sec:descrittori}

Tutto quello che abbiamo detto sta in piedi se disponiamo di alcune informazioni importanti:
\begin{itemize}
\item a che livello è mappato il blocco di dati?
\item a che indirizzo?
\item quanto è grande?
\item è presente o devo trasferirlo?
\end{itemize}
Tutti questi particolari sono ben annotati nei cosiddetti \textit{descrittori}. Nell'architettura IA32 i descrittori hanno una struttura come indicato in figura \ref{fig:descrittoreGenerico}.

\begin{figure}[!h]
\centering
\includegraphics[width=0.75\columnwidth]{img/descrittoreGenerico}
\caption{Struttura generica di un descrittore}
\label{fig:descrittoreGenerico}
\end{figure}

Ad ogni segmento è associato un descrittore di 8 byte contenente tutti gli attributi del segmento (indirizzo di origine, lunghezza del segmento, diritti di accesso, tipo, etc\ldots). Per accedere a un oggetto in memoria è necessario conoscere il segmento di appartenenza; il relativo selettore deve essere caricato in un registro di segmento i quali sono sei\footnote{Quindi in ogni istante il programma ha la visibilità di non più di sei segmenti.}.

In IA32 la dimensione massima di un segmento è 4 GB; all'interno del descrittore sono contenute le informazioni necessarie per verificare se un accesso al segmento è lecito e in tal caso indica dove il segmento si trova.
Il linguaggio macchina delle CPU da 32 bit è un'estensione di quello dei processori 8086/88: così come questi due ultimi processori avevano una memoria segmentata, così anche per l'architettura a 32 bit ad ogni segmento è associato un \textit{descrittore di segmento} (vedi fig. \ref{fig:descrittoreSegDati}). Il descrittore va caricato in un registro di segmento, così come andava caricato in DS l'indirizzo iniziale del segmento in IA16\footnote{Data l'istruzione \textit{assembler} (ALFA è l'offset):\\
\texttt{MOV AL, ES:[ebx + edi].ALFA }\\
L'indirizzo virtuale è calcolato così:\\
\texttt{IV = ES.base\_seg + EBX + EDI + ALFA}\\
In generale, l'indirizzo virtuale (o lineare) di un operando in memoria in IA32 è infatti:\\
\texttt{IV = seg\_reg.base\_seg + EA}}.

\begin{figure}[!h]
\centering
\includegraphics[width=0.75\columnwidth]{img/descrittoreSegDati}
\caption{Descrittore di un segmento dati}
\label{fig:descrittoreSegDati}
\end{figure}

In generale nell'architettura IA32 i descrittori vengono utilizzati non solo per descrivere e gestire i segmenti di codice e dati, ma anche per descrivere e gestire altri tipi di oggetti detti \textit{oggetti di sistema}\footnote{Il bit S\# definito nel byte 5 del descrittore consente di specificare se tale descrittore descrive un oggetto di sistema (S\# = 0) o un oggetto dell'applicazione (S\# = 1).
Gli oggetti dell'applicazione sono suddivisi in:
\begin{itemize}
\item segmenti di codice;
\item segmenti di dati e/o \textit{stack}.
\end{itemize}
Gli oggetti di sistema suddivisi in:
\begin{itemize}
\item segmenti contenenti i descrittori di task detti TSS o \textit{Task
State Segment};
\item porte di accesso a task e procedure di servizio dette \textit{Gates}
(i \textit{Gates} controllano sia l'accesso eseguito a controllo di
programma, sia l'accesso eseguito in risposta a eventi
interni o esterni);
\item tabelle con l'elenco degli oggetti accessibili a un task (\textit{Local
Descriptor Table} o LDT).
\end{itemize}
}

I descrittori sono radunati in apposite tabelle dette \textit{tabelle dei descrittori} (\textit{Global Descriptor Table}, GDT, \textit{Local Descriptor Table}, LDT), come illustrato in figura \ref{fig:sedeDescrittori}. È possibile accedere a un descrittore attraverso una chiave di accesso di 16 bit (detta selettore, la cui struttura è riportata in figura \ref{fig:strutturaSelettori}) contenente sia l'identificatore della tabella (GDT o LDT) sia l'indice del descrittore nella tabella (13 bit). Entrambe le tabelle contengono al massimo 8192 elementi di 8 byte.

\begin{figure}[!h]
\centering
\includegraphics[width=0.75\columnwidth]{img/sedeDescrittori}
\caption{Locazione dei descrittori}
\label{fig:sedeDescrittori}
\end{figure}

\begin{figure}[!h]
\centering
\includegraphics[width=0.75\columnwidth]{img/strutturaSelettori}
\caption{Struttura dei selettori}
\label{fig:strutturaSelettori}
\end{figure}

I descrittori sono parecchi: ve n'è un tipo per i segmenti di dati, uno per i segmenti di codice, uno per i \textit{task}, etc\ldots Per questo esiste anche un identificatore del tipo di descrittore: esso consta di 5 bit, dei quali solo 4 vengono in realtà usati (il primo è fisso ad 1), per un totale di 16 possibili oggetti all'interno della CPU.

\subsection{Descrittore di dato}
\label{sec:descrittoreDato}

\begin{figure}[!h]
\centering
\includegraphics[width=0.75\columnwidth]{img/segDati}
\caption{Descrittore di dato}
\label{fig:segDati}
\end{figure}

In questo descrittore (vedi fig. \ref{fig:segDati}):
\begin{itemize}
\item DPL è il livello di privilegio del segmento associato al descrittore (\textit{Descriptor privilege level});
\item il bit più significativo del campo "'tipo descrittore'" discrimina tra segmenti di dato e segmenti di codice;
\item il bit A (\textit{Accessed}) è resettabile dal software e viene settato automaticamente ogni volta che viene fatto un accesso al segmento (cioè ogni volta che il descrittore viene caricato su un registro di segmento);
\item il bit E (\textit{ExpandDown}) definisce un segmento dati per il quale il limite è l'indirizzo più basso del segmento (segmento di \textit{stack} che cresce verso indirizzi decrescenti);
\item la base del segmento (per esempio l'indirizzo iniziale del segmento in uno spazio di indirizzamento lineare di 4GB) è memorizzato nei byte 2, 3, 4, 7 (vedi fig. \ref{fig:descrittoreGenerico}). Lo spazio di indirizzamento lineare può essere la memoria fisica o la memoria virtuale;
\item il limite del segmento è un campo di 20 bit memorizzato nei byte 0 e 1 e nei bit [0..3] del byte 6 (vedi fig. \ref{fig:descrittoreGenerico}); nei segmenti di codice e nei segmenti dati con E = 0 l'indirizzo base + limite è l'indirizzo lineare più alto del segmento associato al descrittore. Nei segmenti dati con E = 1, invece, sono vietati tutti gli accessi a indirizzi compresi tra base e base + limite;
\item anche se il campo limite è di 20 bit, la dimensione massima di un segmento non è limitata a un MB, ma può essere anche di 4 GB, infatti l'unità di misura di questo campo può essere il byte o la pagina di 4 KB; chi definisce l'unità di misura è il bit G (\textit{granularity});
\item bit G (\textit{Granularity}): se G = 0, il limite è espresso in byte; se G = 1 il limite è espresso in pagine di 4KB;
\item il bit B riguarda la gestione dei segmenti di \textit{stack}: se B = 1 lo \textit{stackpointer} è di 32 bit (ESP) e l'offset massimo è 0F FFFF FFFH, altrimenti lo \textit{stackpointer} è di 16 bit (SP) e l'offset massimo è 0F FFFH;
\item il bit AVL è un bit disponibile (\textit{available}) a chi scrive il sistema operativo;
\item il bit P (\textit{Present}, vedi fig. \ref{fig:regSegmento}, nella \ref{fig:segDati} è ad 1) indica se il segmento è mappato nel livello adiacente della gerarchia delle memorie (cioè se è mappato in memoria virtuale).
\end{itemize}


\subsection{Descrittore di codice}
\label{sec:descrittoreCodice}

\begin{figure}[!h]
\centering
\includegraphics[width=0.75\columnwidth]{img/segCodice}
\caption{Descrittore di codice}
\label{fig:segCodice}
\end{figure}
Oltre ad alcuni attributi illustrati nel paragrafo \ref{sec:descrittoreDato}, abbiamo (vedi fig. \ref{fig:segCodice}).
\begin{itemize}
\item il bit W (\textit{Writable}) stabilisce se un segmento dati è di tipo \textit{Read/Write} o \textit{Read Only};
\item bit D nei segmenti di codice vale 1 se il codice è scritto nel linguaggio macchina della IA-32, mentre vale 0 se si tratta di codice macchina del processore 8086 (in particolare operandi e offset di 16 bit);
\item il bit R (\textit{Readable}) stabilisce se un segmento di codice è di tipo \textit{Execute/Read} o \textit{Execute Only}, cioè se contiene anche dati o è solo eseguibile;
\item il bit C (\textit{Conforming}) di un segmento di codice indica che il livello di privilegio a cui il segmento viene fatto eseguire è quello del codice che ha messo in esecuzione il segmento stesso (il segmento cioè eredita il livello di privilegio del codice chiamante).
\end{itemize}

Si ricorda che il descrittore di codice corrente è contenuto nel registro CS del nostro microprocessore.

\section{Politiche d'avvio e descrittori}
\label{sec:avvioDescrittori}

\begin{figure}[!h]
\centering
\includegraphics[width=0.9\columnwidth]{img/PGeRegistri}
\caption{I registri di controllo della CPU e l'abilitazione dei diversi livelli della gerarchia delle memorie}
\label{fig:PGeRegistri}
\end{figure}

La memoria, all'accensione della macchina, deve partire con i descrittori (e relative tabelle) già pronti: a tal proposito, le tabelle vanno portate nella memoria (vergine) non appena viene fatto partire il nostro dispositivo. Mentre però viene effettuato lo \textit{start-up}, la CPU si comporta come un 8088 (che non aveva i descrittori e quindi può benissimo funzionare senza!) e fa uso di un registro (CR0, vedi figura \ref{fig:PGeRegistri}) che si occupa di gestire il transitorio in cui non si ha alcuna tabella. Il bit PE (\textit{Protection Enable}) è pari a 0 durante tutta la transizione e va ad 1 quando finalmente tutto è pronto per funzionare a regime\footnote{Se PG=0 la memoria virtuale è disabilitata quindi l'indirizzo si riferisce alla memoria fisica. Se PG=1 la memoria virtuale è abilitata e gli indirizzi fanno riferimento alla memoria virtuale (in particolare il registro BASE SEG).}.

Mentre la CPU si accende, la \textit{cache} è disabilitata (CB=1) e anche l'impaginazione (PG=0) lo è: ufficialmente, sotto quest'ultimo punto di vista, tutto funziona come nell'IA16.
Abilitata \textit{cache} e impaginazione, dispongo della gerarchia completa delle memorie.

\section{Ciclo IDLE e considerazioni sui consumi}
\label{sec:idleConsumi}

Il ciclo \textit{idle} (vedi fig. \ref{fig:halt}) è attivo quando la CPU ha ben poco da fare: durante il suo svolgimento, la CPU va in HALT per cercare di spegnere quasi tutto e - di conseguenza - di consumare il meno possibile. I consumi di un moderno elaboratore possono infatti essere suddivisi in tre parti, più o meno equivalenti come dispendio:
\begin{itemize}
\item la CPU;
\item le periferiche (HD, monitor, etc\ldots);
\item tutto il resto (memorie, bridge, etc\ldots).
\end{itemize}

\begin{figure}[!h]
\centering
\includegraphics[width=0.55\columnwidth]{img/halt}
\caption{Struttura del ciclo \textit{idle}}
\label{fig:halt}
\end{figure}

Spegnendo la CPU abbatto i consumi di 1/3; altri accorgimenti per il risparmio dell'energia possono essere l'utilizzo della modalità \textit{standby}, lo spegnimento dell'\textit{hard-disk}, l'inibizione \textit{totale} del clock alla CPU o, infine, la riduzione della frequenza di funzionamento o della tensione\footnote{Servirà ovviamente un sistema di monitoraggio per decidere come modulare questi due parametri in base alla situazione.} in virtù della relazione
\[
P=KV^2f
\]
(con $K$ costante di proporzionalità, $V$ tensione e $f$ frequenza).

Ebbene, quando la CPU è in HALT viene generato un apposito ciclo di bus (che viene visto anche dal bridge) e che fa partire un \textit{timer} il quale, allo scadere, priva la CPU del clock.
Se togliamo anche l'alimentazione, la CPU passa in modalità \textit{deep sleep} (sonno profondo), in cui è il bridge a gestire gli \textit{interrupt}: il risveglio sarà lentissimo, ma almeno avremo risparmiato moltissima energia.


\input{Protezione.tex}
\input{Gerarchia.tex}
\chapter{Struttura della memoria centrale}
\label{cha:centralMem}

\section{Decoder e memorie}
\label{sec:decoderMemorie}

Supponiamo di voler realizzare 1 MB di RAM utilizzando un certo numero di chip: per indirizzare tale quantit� di memoria servono 20 bit, quindi una scelta potrebbe essere quella di utilizzare $2^3$ chip da $2^{20-3}=2^{17}=128K$ byte. Ogni chip avr� perci� la struttura come quella in figura \ref{fig:decoderino}, con 17 bit di indirizzamento e un segnale di \textit{enable} per attivare o meno il dispositivo (e spegnerlo quando non c'� bisogno, per evitare conflitti e risparmiare preziosa energia).

\begin{figure}[!h]
\centering
\includegraphics[width=0.65\columnwidth]{img/decoderino}
\caption{Chip da 128 KB (17 bit di indirizzamento)}
\label{fig:decoderino}
\end{figure}

All'interno del chip � presente un decoder, dispositivo a $N$ ingressi e $2^N$ uscite, una sola delle quali � attiva in un certo istante: esiste infatti una sola uscita ad '1', qualsiasi sia la configurazione d'ingresso.

Volendo, possiamo sfruttare la propriet� associativa della moltiplicazione [$x_1x_2x_3=(x_1x_2)x_3$] per costruire un decoder a $2^N$ uscite a partire da $2^{(N-K)}$ analoghi componenti con $2^K$ uscite: nell'esempio in figura \ref{fig:decoderone} si mostra come ottenere un decoder a $2^4=16$ uscite a partire da $2^{(4-2)}=4$ decoder a $2^2=4$ uscite.

\begin{figure}[!h]
\centering
\includegraphics[width=0.55\columnwidth]{img/decoderone}
\caption{Costruzione di un decoder a $2^N$ uscite a partire da $2^{(N-K)}$ analoghi componenti con $2^K$ uscite}
\label{fig:decoderone}
\end{figure}

Il decoder \textit{root}, cio� quello di livello pi� alto, utilizza i due bit pi� significativi ($x_3x_2$) per discriminare quale dei quattro decoder "'figli'" attivare tramite il segnale di \textit{enable}: ognuno di questi, a sua volta, sceglie quale uscita far passare ad '1' in base ai bit meno significativi $ x_1x_0$.
Nell'esempio fatto ad inizio capitolo (1 MB RAM, 8 chip da 128 KB) possiamo scegliere di usare 3 bit per il decoder \textit{root} e 17 bit per ogni chip come quello in figura \ref{fig:decoderino}.

\section{Bus dati con parallelismo di $n$ byte}
\label{sec:multipliByte}

\textsf{
\textit{NOTA PRELIMINARE: in questo paragrafo faremo due esempi. Il primo, con 4 banchi (v. oltre), si riferisce ad una  architettura a 32 bit. Il Pentium, con bus dati a 64 bit, dispone di 8 banchi per poter sfruttare al massimo il suo parallelismo.}} \\


La quantit� minima di informazioni normalmente indirizzabile da parte di un microprocessore � pari ad un byte, ma non � necessario che il bus dati del microprocessore abbia il parallelismo limitato a un byte. Ci si pu� quindi porre l'obiettivo di aumentare il parallelismo del bus al fine di trasferire pi� byte in un solo trasferimento, col vincolo di conservare l'indirizzabilit� del singolo byte. Vogliamo cio� aumentare il \textit{throughput} del bus senza cambiare le temporizzazioni, portando il parallelismo del bus dati a $m$ byte, con $ m = 2^k$, in modo da poter trasferire fino a $2^k$ byte per ogni trasferimento. Tutto ci� � arduo da farsi all'interno di una memoria con struttura monolitica. Piuttosto, fa maggiormente al caso nostro una memoria suddivisa in pi� "'compartimenti'" (banchi) in grado di essere attivi indipendentemente l'uno dall'altro (a differenza di quanto permette il decoder, che accende sempre una sola uscita alla volta). Per realizzare quanto appena detto � necessario che indirizzi consecutivi appartengano a chip diversi (\textit{interleaving}); si osservi ad esempio la figura \ref{fig:bancobanchi}: essa mostra come sia possibile suddividere una memoria da 4 GB in pi� banchi di memoria, ognuno di dimensione inferiore (2x2 GB, 4x1 GB). Se ci raffiguriamo mentalmente la nostra memoria bidimensionale (il rettangolone con gli indirizzi scritti di fianco, tanto per intenderci) dobbiamo immaginare l'indirizzamento a partire da in basso a destra e andando prima verso sinistra e poi verso l'alto, come suggerisce la figura \ref{fig:alfaBetaBanchi}, con il primo byte nel banco 0 avente indirizzo 0, il secondo avente indirizzo 4, il terzo avente indirizzo 8 etc\ldots (se leggiamo dal basso verso l'alto, nel banco 1 avremo invece gli indirizzi 1, 5 e 9\ldots, nel banco 2 gli indirizzi 2, 6, 10\ldots e cos� via).

\begin{figure}[!h]
\centering
\includegraphics[width=0.5\columnwidth]{img/bancobanchi}
\caption{Frammentazione della memoria monoblocco in banchi di varie dimensioni}
\label{fig:bancobanchi}
\end{figure}

Se, come abbiamo detto poco fa, � possibile attivare i banchi indipendentemente l'uno dall'altro, sar� possibile selezionare fino a 4 byte (32 bit) a seconda se vogliamo un byte, una \textit{word} (2 byte, 2 banchi attivi), una\textit{ double word} (4 byte, 4 banchi attivi). Nell'esempio in figura \ref{fig:alfaBetaBanchi}, possiamo prendere ALFA selezionando il banco 0, mentre BETA � stata allocata ad indirizzi contigui (1-4) secondo la convenzione \textit{little endian}.

\begin{figure}[!h]
\centering
\includegraphics[width=0.65\columnwidth]{img/alfaBetaBanchi}
\caption{Divisione della memoria in banchi e allocazione delle variabili}
\label{fig:alfaBetaBanchi}
\end{figure}

I segnali che permettono di abilitare i banchi e distinguere fra i vari casi descritti sopra vengono detti \textit{bank enable} (BE). Per quanto detto fin'ora, possiamo sinteticamente affermare che il \textit{chip select} (CS) di un qualsiasi blocco di memoria mappato nella finestra di indirizzamento di 4 GB del nostro calcolatore sar� individuato da due coordinate: la finestra di offset (coordinata verticale), che si configura come il classico \textit{chip select} visto a Calcolatori L-A, e il \textit{bank enable} (BE). Il microprocessore sa dove comincia un determinato dato, quale sia la sua dimensione e il suo offset nel banco dunque � perfettamente in grado di accedervi.

\begin{figure}[!h]
\centering
\includegraphics[width=0.75\columnwidth]{img/microMultiByte}
\caption{Struttura del bus degli indirizzi in un microprocessore con
bus dati e indirizzi di 32 bit}
\label{fig:microMultiByte}
\end{figure}

In figura \ref{fig:microMultiByte} notiamo che:
\begin{itemize}
\item i pin della CPU da interfacciare al bus dati sono suddivisi in gruppi di 8 pin (chiamati BD[\ldots]);
\item ad ogni gruppo � associato un segnale $BEi$: se � attivo, la CPU trasferisce dati sul corrispondente insieme di pin BD[\ldots];
\item ad ogni ciclo di bus il microprocessore genera un offset e attiva uno o pi� segnali di \textit{bank enable};
\item ogni \textit{bank enable} attivo individua una delle quattro celle da un byte associate all'offset presente sul bus degli indirizzi in posizione A[31..2]: questo byte sar� trasferito su un ben preciso byte del bus dati. Chiaramente il numero dei  \textit{bank enable} attivi dipende dal numero di byte del dato da trasferire;
\item se il dato da trasferire � composto da byte con diversi valori di offset (dato non allineato), il
microprocessore non ha scelta e dovr� effettuare pi� trasferimenti sul bus; in caso contrario (dato allineato) il trasferimento avviene solamente in un ciclo.
\end{itemize}

Le seguenti sono le formule per la traduzione dell'indirizzo dato in 2 coordinate e quello fisico (cio� preso "'in senso classico'") e viceversa:
\[
\begin{gathered}
IF = \text{Offset} \cdot Numero_{banchi} + BankId \\
BankId = \operatorname{MOD_{NumeroBanchi}}(IF) \\
\text{Offset} = \dfrac{IF}{Numero_{banchi}} = IF >> \log_2\left( Numero_{banchi}\right)
\end{gathered}
\]

\begin{figure}[!h]
\centering
\includegraphics[width=0.85\columnwidth]{img/3234bit}
\caption{Configurazione di offset e BE nel caso di bus dati da 32 bit}
\label{fig:3234bit}
\end{figure}

In figura \ref{fig:3234bit} viene mostrata la corrispondenza fra l'array di 32 bit che indirizza i 4 GB "'in senso classico'" e la rappresentazione a 34 bit (30 di offset + 4 \textit{bank enable}) del nostro sistema a \textit{bus multibyte}. Chiaramente i 34 bit non possono viaggiare in un bus degli indirizzi da 32 bit: per questo � presente un encoder in grado di tradurre l'array dei \textit{bank enable} BE[3..0] in un numero binario di 2 bit. Allo stesso modo, nel nostro Pentium con bus dati a 64 bit (e bus degli indirizzi a 32 bit), abbiamo bisogno di un encoder che prenda in pasto gli 8 \textit{bank enable} (ogni banco occupa 512 KB) e lo traduca nei bit\footnote{Non posseduti dal microprocessore.} A2, A1 e A0 da inserire nel "'classico'" indirizzo da 32 bit (quello che daremmo in una memoria monobanco) come mostrato in figura \ref{fig:banktraduzione64}.

\begin{figure}[!h]
\centering
\includegraphics[width=0.5\columnwidth]{img/banktraduzione64}
\caption{Traduzione dei segnali di \textit{bank enable} nel caso a 8 banchi}
\label{fig:banktraduzione64}
\end{figure}

\section{E le periferiche?}
\label{sec:perifericheCacchio}

Per quanto riguarda le periferiche, � il \textit{bridge} che si occupa di intercettare pi� banchi e interfacciare correttamente il bus di I/O (8 bit) e quello dei dati (64 bit, sempre nel Pentium).
Nei trasferimenti da bus dati a bus di input/output il bridge funge infatti da multiplexer da 8 bit a 8 vie, mettendo cio� in comunicazione ciascuno degli 8 banchi del bus dati con il bus di I/O. Nei trasferimenti in direzione opposta (da bus di input/output a bus di memoria) il bridge trasferisce il contenuto del bus di I/O sui banchi del bus dati. Di conseguenza, grazie al bridge, la CPU pu� comunicare con il bus di I/O attraverso qualunque dei suoi banchi del bus dati: questo significa che i dispositivi interfacciati al bus di I/O possono essere mappati a qualunque indirizzo, quindi in particolare a indirizzi adiacenti. Inoltre � possibile effettuare trasferimenti in DMA di blocchi di dati contigui (come vedremo in seguito).

Per mappare le periferiche si segue dunque lo schema d'indirizzamento riportato in figura \ref{fig:perifericheBanchi}.

\begin{figure}[!h]
\centering
\includegraphics[width=0.85\columnwidth]{img/perifericheBanchi}
\caption{Gestione degli indirizzi in un sistema con bus dati da 64 bit e bus di I/O da 8 bit}
\label{fig:perifericheBanchi}
\end{figure} 

Si noti che � indispensabile porre ogni chip ad un indirizzo allineato (cio� multiplo della sua dimensione).

\section{Un esempio}
\label{sec:esempio}

Un sistema A basato su una CPU da 32 bit, con bus dati da 64 bit e un bus di I/O da 8 bit (separati
da un bridge), dispone di un DMAC, un PIC e una porta seriale S1 (un chip 8250).

\begin{figure}[!h]
\centering
\includegraphics[width=0.95\columnwidth]{img/esempioPerifericheMappate}
\caption{Posizione dei dispositivi di Input/Output nello spazio di indirizzamento di I/O}
\label{fig:esempioPerifericheMappate}
\end{figure}

In figura \ref{fig:esempioPerifericheMappate} viene mostrata una possibile scelta di \textit{mapping} delle periferiche in I/O. Si noti la presenza di 8 banchi (ognuno da 8 KB per un totale di 64 KB di spazio di indirizzamento in I/O). A 100H � stato mappata la seriale (8 byte, si veda la figura \ref{fig:mux8vie} per uno "'zoom'" dello schema d'interfacciamento fra essa e la memoria centrale tramite il bridge), a 200H il PIC (2 byte) e a 300H il DMA Controller (16 byte): tutti gli indirizzi sono quindi allineati!
Volendo, ad esempio, accedere al registro IER, l'indirizzo sar�:
\begin{verbatim}
A15 A14 A13 A12 | A11 A10 A9 A8 | A7 A6 A5 A4 | A3 A2 A1 A0
  0   0   0   0 |   0   0  0  1 |  0  0  0  0 |  0  0  0  1

In esadecimale: 0101F
\end{verbatim}

\begin{figure}[!h]
\centering
\includegraphics[width=0.75\columnwidth]{img/mux8vie}
\caption{Schema di interfacciamento fra la seriale e la memoria centrale tramite il bridge}
\label{fig:mux8vie}
\end{figure}

\chapter{Il bridge}
\label{cha:bridge}

\section{Generalità e piccola digressione sul MUX}
\label{sec:generalitaDMAC}

Il DMA Controller fornisce un meccanismo d'accelerazione per il trasferimento dei dati fra la RAM e la periferica; il suo scopo è quello di fungere da co-processore di supporto, assumendo di tanto in tanto il ruolo di \textit{master} del bus, alleggerendo la CPU e diminuendo drasticamente il numero di interrupt che quest'ultima deve gestire. La presenza contemporanea di CPU e DMAC rende la nostra architettura già \textit{multiprocessor}, almeno a livello concettuale.

\subsection{Il MUX e altri animali}

\begin{figure}[!h]
\centering
\includegraphics[width=0.58743626456\columnwidth]{img/mux2vie}
\caption{Radiografia di un multiplexer a 2 vie}
\label{fig:mux2vie}
\end{figure}

In figura \ref{fig:mux2vie} vediamo schematicamente l'interno di un semplicissimo multiplexer a 2 vie\footnote{Notiamo la presenza del fondamentale componente \textit{Buffer 3-state} e di un gate NOT.}:
\[
u=a \bar c + bc
\]
In figura \ref{fig:mux4vie} i vediamo invece la struttura di un multiplexer a 4 vie costruito mediante un decoder 2:4 (che ci dà la sicurezza di avere un'unica via attivata).

\begin{figure}[!h]
\centering
\includegraphics[width=0.57676586356213452354687654324567876543\columnwidth]{img/mux4vie}
\caption{Radiografia di un multiplexer a 4 vie}
\label{fig:mux4vie}
\end{figure}

\section{Bridge e sua struttura}
\label{sec:bridgeStruttura}

Se al posto dei buffer ora mettiamo i componenti 244 ("'batterie'" di 8 buffer divise in due gruppi da 4, attivate dagli \textit{output enable }OE0 ed OE1, vedi fig. \ref{fig:244}) siamo già sulla buona strada per creare una delle due direzioni del nostro \textit{bridge}: in parole povere esso funziona come un multiplexer a 8 vie, solo che invece si convogliare i singoli bit si convogliano blocchi da 8 bit (vedi fig. \ref{fig:briggiu}).

\begin{figure}[!h]
\centering
\includegraphics[width=0.35\columnwidth]{img/244}
\caption{Il componente 244}
\label{fig:244}
\end{figure}

\begin{figure}[!h]
\centering
\includegraphics[width=0.75\columnwidth]{img/briggiu}
\caption{Bridge: dalla memoria alle periferiche}
\label{fig:briggiu}
\end{figure}

Al decoder in figura \ref{fig:briggiu} arrivano i tre bit generati dall'encoder che prende in pasto gli 8 segnali di \textit{bank enable} (vedi capitolo \ref{cha:centralMem}): a questo punto qualcuno potrebbe obiettare che non ha senso mettere un decoder a valle di un encoder, tuttavia è opportuno che siano presenti entrambi in quanto c'è il rischio che non vi sia sempre un unico segnale di \textit{bank enable} attivo, cosa che potrebbe portare ad un conflitto elettrico all'interno del MUX. Il nostro scopo è invece quello di svincolare il microprocessore dal doversi preoccupare dei conflitti del multiplexer\footnote{Magari chi ha scritto il software era uno stolto!}, per cui rinunceremo al risparmio che otterremmo nel rimuovere la cascata encoder + decoder.

Problema: stanti così le cose, il bus di I/O rischia di essere contemporaneamente pilotato sia dalla periferica che da i driver 244. Anche questo potrebbe essere causa di conflitto elettrico, dunque dobbiamo tirare in ballo qualche segnale di controllo (IORD*, IOWR*, MEMRD*, MEMWR*)\footnote{La dicitura potrebbe cambiare, nel corso della trattazione: si potrà ad esempio trovare IO/MEM*, W/R*, MRDC*, IOWC*, etc\ldots Stiamo comunque parlando degli stessi segnali, l'importante è sapersi intendere!} per segnalare se la CPU sta leggendo (RD) o scrivendo (WR) in memoria (MEM) o da/verso le periferiche (IO). Se ad esempio stiamo scrivendo sulla periferica sarà necessario abilitare il bridge tramite l'abbassamento del segnale IOWR*. Risulta inoltre intuitivo il fatto che il DMAC, effettuando trasferimenti fra la periferica e la memoria, dovrà in tali frangenti attivare contemporaneamente MEMWR* e IORD*.

Sarà inoltre necessario contemplare uno schema come quello in figura \ref{fig:briggiu}, che "'salga'" verso le memorie compiendo il percorso inverso rispetto a quello visto poco fa. Nella sostanza nulla cambia, basta girare le frecce: l'unico punto delicato consiste nella scelta dei segnali di \textit{output enable } per i 244 che portano al bus della memoria. In questo caso non c'è bisogno di un decoder per pilotarli: possiamo anzi attivarli tutti contemporaneamente (quando leggiamo dalla periferica), tanto poi il microprocessore aggiornerà soltanto il banco coinvolto in quel particolare ciclo di bus e in quella particolare transazione (gli altri manco li degna di attenzione!).
Quel che non deve accadere è che vi siano conflitti fra la lettura e la scrittura, quindi dovremo oportunamente collegare i segnali MEM/IO* e WR/RD* agli \textit{output enable} di cui sopra: in questo modo il nostro multiplexer si spegnerà quando non ci sarà bisogno di lui, con un conseguente risparmio anche in termini energetici e, allo stesso tempo, eviteremo l'eventuale conflitto che si potrebbe avere quando il microprocessore va a leggere in memoria e attiva RD* (se non vi fosse MEM/IO* il bridge si attiverebbe senza motivo, visto che leggiamo in memoria e non nella periferica).

Se siamo sadici e preferiamo interfacciare la nostra periferica direttamente al bus dati\footnote{Scelta sconsigliata!} (piuttosto che sfruttare il bridge), dobbiamo avere cura di predisporre un segnale che disabiliti il bridge quando leggiamo da quella determinata periferica: in caso contrario vi sarebbe un conflitto elettrico fra il "'sopra'" (memoria, CPU e a questo punto anche la nostra periferica) e il "'sotto'" (DMAC, PIC, altre periferiche, etc\ldots) del nostro famoso schema (riportato anche in figura \ref{fig:schemaSolito}). Tutto questo comporta un complicarsi della logica di controllo e richiede maggiore attenzione in sede di progetto.

Regola generale è comunque la seguente: se interfaccio il dispositivo in memoria userò indirizzi \textit{interleaved}, mentre se collego la periferica al bus di I/O gli indirizzi saranno contigui.

\begin{figure}[!h]
\centering
\includegraphics[width=0.75\columnwidth]{img/schemaSolito}
\caption{Struttura del calcolatore (in sitesi)}
\label{fig:schemaSolito}
\end{figure}

In figura \ref{fig:ciclivari} vengono graficate alcune forme d'onda rappresentanti i segnali che qualificano il ciclo di bus: se M/IO* è vero e W/R* è negato significa che stiamo leggendo dalla memoria (è l'equivalente del segnale MEMRD*, chiamato MRDC* in figura); se, viceversa, M/IO* è negato e W/R* è vero significa che stiamo scrivendo in memoria (IOWRC* attivato), ad esempio perché stiamo effettuando un'operazione di OUT. Si noti che c'è bisogno di rispettare un tempo di \textit{set-up} e di \textit{hold} perché il dato sia valido. 

\begin{figure}[!h]
\centering
\includegraphics[width=0.6\columnwidth]{img/ciclivari}
\caption{Segnali che qualificano il ciclo di bus}
\label{fig:ciclivari}
\end{figure}

Qualificare il ciclo di bus è importante non solo per gli ovvi motivi riguardanti il bridge e i conflitti elettrici, ma anche perché in base a cosa dovremo fare avremo un ciclo di bus specifico fissato da una macchina a stati (vedi fig. \ref{fig:macchinaStatiCicloBus}.

\begin{figure}[!h]
\centering
\includegraphics[width=0.65\columnwidth]{img/macchinaStatiCicloBus}
\caption{Macchina a stati che regola il ciclo di bus}
\label{fig:macchinaStatiCicloBus}
\end{figure}

\section{Bridge bidirezionale}
\label{sec:bridgeBidi}

Se utilizziamo dei componenti 245 siamo in grado di creare un \textit{bridge bidirezionale} che sappia sfruttare i segnali di qualificazione del ciclo di bus per una gestione ottimale del dispositivo. Due sono le vie che è possibile seguire (e tre gli stati possibili):
\begin{itemize}
\item "'verso l'alto'", cioè dalle periferiche alla memoria (chiameremo questo caso \textit{ricezione});
\item "'verso il basso'", cioè dalla memoria alle periferiche (chiameremo questo caso \textit{trasmissione});
\item in alternativa a queste opzioni è possibile spegnere completamente il dispositivo (non dobbiamo far nulla).
\end{itemize}
Per gestire queste tre eventualità servono i segnali di controllo:
\begin{itemize}
\item T/R*, di trasmissione/ricezione;
\item OE*, di \textit{output enable}, che indica se vogliamo fare qualcosa oppure no.
\end{itemize}

\begin{figure}[!h]
\centering
\includegraphics[width=0.55\columnwidth]{img/bridgeBidirezionale}
\caption{Bridge bidirezionale costruito con componenti 245}
\label{fig:bridgeBidirezionale}
\end{figure}

A cosa colleghiamo T/R* e OE*?
T/R* sarà l'OR fra il segnale IORD* (che da solo non basterebbe, in quanto li PIC potrebbe intervenire quando più gli aggrada e confliggere) e INTA*. OE* lo colleghiamo invece al segnale M/IO*, tanto il bridge dev'essere non attivo tutte le volte che il microprocessore va in memoria e non sta andando in IO.

\input{DMA.tex}
\chapter{Esercizi: Tomasulo}
\label{sec:tomasulo}

In questo capitolo esporremo qualche consiglio per la risoluzione del tipico esercizio 2 (su Tomasulo).

\section{Dinamica dell'esecuzione}
\label{sec:dinamicaTomasulo}

Grazie all'approccio di Tomasulo � possibile risolvere tutti i problemi relativi alle alee. Fortunatamente, visto l'algoritmo stesso e la struttura del DLX, dobbiamo preoccuparci solo delle alee RAW. 
Prendiamo, come esempio, il seguente esercizio:\\

\textsf{Un DLX con Tck = T dispone di tre unit� funzionali multiciclo, A, B e C, capaci di eseguire le seguenti istruzioni su operandi in virgola mobile:}
\begin{verbatim}
FDIV F1, F2, F3
FADD F7, F1, F4
FMUL F2, F1, F2
FADD F5, F1, F6
FADD F1, F3, F4
FMUL F8, F9, F10
\end{verbatim}
\textsf{I tempi di esecuzione di ciascuna unit� funzionale sono di 2T per le FADD, 3T per le FMUL, 5T per le FDIV.}

Mostriamo ora la dinamica dell'esecuzione nel caso di 2 \textit{reservation station} per unit� funzionale (si faccia riferimento alla figura \ref{fig:tomasEs1}.
\begin{figure}[!h]
\centering
\includegraphics[width=\columnwidth]{img/tomasEs1}
\caption{Dinamica dell'esecuzione nel caso di 2 \textit{reservation station} per unit� funzionale.}
\label{fig:tomasEs1}
\end{figure}
Indichiamo:
\begin{itemize}
\item con IF la fase di \textit{fetch};
\item con ID la fase di \textit{decode};
\item con le terne YXZ la \textit{reservation station} di indice X riservata all'operazione di ADD (Y = A), MUL (Y = B), DIV (Y = C), mentre siamo in attesa dell'operando (Z = W = WAIT), oppure quando l'operando � pronto ma non possiamo calcolare perch� siamo privati dell'unit� funzionale (Z = R = READY) o infine mentre stiamo effettuando la nostra operazione (Z = E = EXECUTE). Ad esempio: A0E = \textit{reservation station} 0 riservata alla ADD, stiamo calcolando la somma (E);
\item con WB la fase di \textit{write back}.
\end{itemize}

Una prima cosa che possiamo notare � che ogni istruzione entrante nella \textit{pipeline} deve anzitutto effettuare la \textit{fetch} (diagonale rossa). Ci� � sempre vero, qualsiasi sia lo stato delle alee o il numero delle istruzioni in gioco: non � per� sempre detto che la \textit{fetch} duri un unico clock o che sia seguita immediatamente la fase di \textit{decode}. L'ultima ADD, infatti, � obbligata a stallare in ID in quanto non vi sono \textit{reservation station} libere (A0 � occupata e siamo in attesa di ricevere l'operando F1 dalla prima DIV per poter eseguire F1 + F4 => F7; anche A1 � occupata perch� sta attendendo F1): se avessimo avuto tre RS per ogni unit� funzionale (v. oltre) l'ultima ADD non avrebbe dovuto stallare in ID ma avrebbe potuto immediatamente procedere con una A2E. Notiamo inoltre che l'ultima istruzione (MUL) rimane in \textit{fetch} intanto che la ADD immediatamente precedente stalla in ID per carenza di RS (\textit{alea strutturale}).
Torniamo in cima: la DIV � la prima istruzione � pu� immediatamente partire, perci� la sua evoluzione temporale � ovvia (IF, ID, 5 cicli per effettuare l'operazione, WB). La prima istruzione, qualsiasi sia l'esercizio, pu� sempre andare immediatamente in EXECUTE, non dovendo attendere operandi da altre istruzioni.
La seconda, purtroppo, non pu� fare altrettanto: F1, infatti, deve prima essere aggiornato dall'istruzione precedente, eventualit� che si verifica a seguito della fase di WB della DIV (in T8). Notiamo infatti che si ha A0W (attesa di operandi) da T4 fino a T8 e quindi A0E per i due cicli successivi. Attenzione: la fase di EXECUTE comincia sempre il ciclo successivo la WB dell'istruzione precedente che ha aggiornato l'operando!
Nel frattempo anche la MUL � costretta ad aspettare il WB della DIV (T8) per poter partire: l'evoluzione di tale MUL si snoda quindi in modo assolutamente analogo alla ADD precedente. Si noti che � possibile effettuare parallelamente una ADD, una MUL e una DIV (e infatti in T9 e T10 abbiamo due fasi E contemporanee), ma � assolutamente impossibile effettuare simultaneamente due operazioni dello stesso tipo: una buona verifica della bont� della soluzione � quindi quella di controllare che non vi siano mai pi� di una ADD o pi� di una MUL o pi� di una DIV contemporaneamente in esecuzione!
Dopo la MUL, ecco un'altra ADD: siamo salvi, abbiamo infatti occupato solo una delle due RS disponibili! In T6, quindi, possiamo scrivere A1W e metterci nuovamente in attesa dello stramaledetto F1 in uscita dalla DIV. Quel che non possiamo fare � invece eseguire la ADD in T9: gi� una ADD � infatti in esecuzione (A0E), quindi l'unica � riparare su A1R (READY: l'operando F1, richiesto, � pronto, ma non possiamo ancora eseguire a causa di \textit{alea strutturale}).
T10 � l'ultimo ciclo di clock in cui l'unit� funzionale delle ADD � occupata: in T11, quindi, A1R pu� passare in A1E.
L'ultima ADD � costretta a stallare in ID, come gi� abbiamo sottolineato: ma non facciamoci ingannare! Quand'anche la \textit{reservation station} A0 si libera non possiamo fiondarci ad assegnare un identico \textit{tag} alla nostra nuova istruzione! Prima dobbiamo effettuare la "'vera'" ID (in T12, "'causata'" dal WB nel precedente T11) e quindi solo in in T13 possiamo scrivere A0E anche per la nostra ADD.
L'ultima MUL rimane invece in IF intanto che la ADD precedente stalla in ID; in T13 arriva il turno della MUL e infine si hanno tre cicli di esecuzione (B0E) prima dell'ultimo WB. \\

\begin{figure}[!h]
\centering
\includegraphics[width=\columnwidth]{img/tomasEs1b}
\caption{Dinamica dell'esecuzione nel caso di 3 \textit{reservation station} per unit� funzionale.}
\label{fig:tomasEs1b}
\end{figure}

Se disponiamo di 3 RS (invece che 2) per unit� funzionale, risparmiamo qualche \textit{clock} rispetto al caso precedente, in quanto possiamo evitare l'alea strutturale dell'ultima ADD. In figura \ref{fig:tomasEs1b} si mostra tale risultato: si noti che i clock totali sono 14 invece che 17, per uno \textit{speed-up} pari a 
\[
\dfrac{17}{14} = 1,214
\]
Il risparmio � dovuto principalmente alla velocizzazione delle ultime due istruzioni, come gi� anticipato: in particolare, l'ultima ADD pu� sfruttare un vuoto della relativa unit� funzionale, la quale non � utilizzata dalle altre due analoghe istruzioni a causa della mancanza di F1 (che ci tocca sorbire fino a F8). 
Si noti infine che, in caso di pi� WB contemporanee, � necessario far progressivamente stallare quelle delle istruzioni successive alla prima, nonch� all'unica, istruzione che pu� davvero eseguire il WB\footnote{Si veda tuttavia il paragrafo \ref{sec:casiParticolari} per maggiori chiarimenti.}. Nell'esempio, il \textit{clock} incriminato � T11: la ADD sta effettuando il WB e l'ultima MUL � costretta a ripeterlo in T12.

\section{Film delle variazioni dei registri}
\label{sec:filmRegistri}

\textsf{Si mostri il film delle variazioni dei registri F1 e F8 immaginando, per semplicit�, che all'inizio si abbia $Fi=i$.}

\begin{figure}[!h]
\centering
\includegraphics[width=\columnwidth]{img/tomasEs1c}
\caption{Film delle variazioni dei registri}
\label{fig:tomasEs1c}
\end{figure}

Mostrare il film dei registri significa illustrare l'evoluzione della coppia di valori Q e V per i registri richiesti. Il parametro V � il valore assunto da un determinato registro in un particolare momento, mentre Q pu� essere pari a 0 (se il valore V � quello attualmente valido) o pari all'identificatore della RS che "'restituir�'" tale risultato. Se Q � pari all'identificatore di una certa RS, allora significa che potremo porre Q = 0 e V pari al vero valore che il registro avr� in tale momento quando l'istruzione che ha prenotato quella \textit{reservation station} avr� effettuato il WB (o, pi� precisamente, il ciclo successivo). 
Esaminiamo la figura \ref{fig:tomasEs1c} e il registro F1: inizialmente la coppia (Q,V) sar� (0,1) in quanto il valore di partenza del registro F1 � (per ipotesi) 1 ed � un valore gi� disponibile e, per cos� dire, "'aggiornato'". Ogni volta che un'operazione prenota una \textit{reservation station} per il calcolo del valore di F1 dobbiamo impostare nuovamente il parametro Q: cos� ad esempio avviene in T3 (la DIV passa in C0E e quindi Q = C0) e in T7 (la ADD va in A2E e quindi Q = A2). Il valore V, in entrambi questi casi, non va toccato: esso va invece aggiornato nel ciclo successivo alla WB associata alla \textit{reservation station} corrispondente al valore Q. Nel nostro caso, Q = A2, quindi dobbiamo aggiornare V il ciclo successivo a quello in cui l'ultima ADD (quella che ha prenotato la \textit{reservation station} A2) ha effettuato il WB (cio� in T10, quando F1 diventa pari a 3 + 4 = 7).

Detto questo, il film di F8 diventa una banalit�: la prima volta in cui un'istruzione bussa ad una RS per F8 avviene in T8, quando la MUL prenota la \textit{reservation station} B1: in T8 andremo quindi a scrivere Q = B1. Se nessun'altra istruzione vuole scrivere su B8 (e, per fare ci�, prenota una RS), dovremo aggiornare il valore V di F8 il ciclo successivo in cui la MUL avr� eseguito il WB. Quanto detto � effettivamente ci� avviene: in T12 la MUL effettua il WB e in T13 il valore del registro viene aggiornato.

\section{Film delle variazioni delle RS}
\label{sec:filmRS}

\textsf{Si mostri il film delle variazioni della RS B0 nel caso di 3 RS per unit� funzionale, immaginando per
semplicit� che all'inizio si abbia $Fi=i$.}\\

\begin{figure}[!h]
\centering
\includegraphics[width=\columnwidth]{img/tomasEs1d}
\caption{Film delle variazioni delle \textit{reservation station}}
\label{fig:tomasEs1d}
\end{figure}

La metodologia e l'approccio non � tanto diverso rispetto al caso precedente: in ogni RS sono presenti due coppie (Q,V), riservate ciascuna ad un operando dell'operazione che prenoter� la RS. I valori di Q e V all'interno della RS variano con le stesse modalit� descritte nel paragrafo \ref{sec:filmRegistri}.
Nel nostro caso si chiede di disegnare il film della RS B0: la prima istruzione a prenotarla � la \{MUL F1, F2, F1\} quindi dovremo andare a scrivere le coppie (Q,V) di F2 e F1. F2 non � mai stato toccato quindi gli assegneremo il valore di \textit{default}, cio� 2, scrivendo la coppia (0,2); F1, invece, � il fastidiosissimo registro monopolizzato dall'altrettanto fastidiosa DIV, la quale ha prenotato la RS C0 in T3. La coppia (Q,V) sar� quindi (C0,1) e varier� in (0,2/3) il ciclo successivo al WB della DIV. La \textit{reservation station} B0 potr� quindi essere disimpegnata il ciclo successivo al WB dell'istruzione che l'ha prenotata (la MUL: WB in T14 e rilascio in T15).

\section{Miglioramento delle prestazioni: calcolo dello \textit{speed-up}}
\label{sec:viagra}

\textsf{Si calcoli lo \textit{speed up }di entrambe le soluzioni rispetto al caso di una sola unit� funzionale capace di
eseguire le tre istruzioni e si commenti il risultato}.\\

Nel caso di una sola unit� funzionale non in pipeline il numero di \textit{clock} necessari � dato dalla semplice somma della durata di tutte le istruzioni (da considerarsi come eseguite in maniera sequenziale). Nel nostro caso � data da: 
\begin{itemize}
\item 1 clock per la IF
\item 1 clock per la ID
\item 2 clock per ogni ADD (per un totale di 6)
\item 5 clock per ogni DIV (per un totale di 5)
\item 3 clock per ogni MUL (per un totale di 6)
\item 1 clock per il WB
\end{itemize}
Il totale restituisce 20 clock, risultato piuttosto modesto confrontato coi 17 e i 14 clock che si hanno nel caso di due e tre RS per ogni unit� funzionale. L'incremento delle prestazioni dovuto all'inserimento delle RS pu� essere quantificato tramite il parametro \textit{speed-up} calcolabile semplicemente come rapporto fra i clock della caso pi� "'lento'" rispetto a quelli del caso pi� "'veloce'" (cio� con apportato il miglioramento in questione). \\
\textit{Speed up} con 2 RS: 
\[
\dfrac{20}{17}=1,176
\]
\textit{Speed up} con 3 RS:
\[
\dfrac{20}{14}=1,429
\]

\section{Casi e domande particolari}
\label{sec:casiParticolari}

\begin{itemize}
\item Nel caso di istruzioni che tentino di effettuare contemporaneamente un WB, abbiamo detto nel paragrafo \ref{sec:dinamicaTomasulo}, le istruzioni "'successive'" alla prima che riesce ad effettuare il WB dovranno ripetere tale fase per poterla effettivamente compiere. S'intende per� che ha la precedenza chi, per prima, � andata in \textit{execute} e pu� accadere che un'istruzione dichiara diverse "'posizioni'" sotto un seconda istruzione sia in grado di effettuare il WB prima di quest'ultima. 
Ad esempio:
\begin{verbatim}
ADD   F18, F1, F22       IF  ID  ...  ...  A0W  A0E  ...  ...  WB   WB <- !!
...
...
MUL   F19, F3, F2        ... IF  ID   B0E  B0E  ...  ...  ...  WB  
\end{verbatim}
\item Se si chiede di provare a cambiare l'ordine delle istruzioni, e di verificare possibili aumenti prestazionali dovuti ad un conseguente risparmio di cicli di \textit{clock}, bisogna stare attenti a non farsi prendere dalla mano e verificare che il nuovo ordine non alteri il valore dei registri. Non � la stessa cosa fare $(1+2)\cdot 3$ piuttosto che $1 + (2\cdot 3)$!
\item Se si chiede di calcolare il minimo numero di cicli di \textit{clock}, qualunque sia il numero di RS, CRB e stadi di \textit{fetch} e \textit{decode} disponibili, basta effettuare il procedimento seguente: si calcola il massimo fra i seguenti prodotti
\[
\text{Numero di istruzioni di tipo X} \cdot \text{Clock impiegati per l'esecuzione dell'istruzione X}
\]
Quindi, se la ADD impiega 3 clock (4 ADD da effettuare), la MUL impiega 4 clock (2 MUL da effettuare), la DIV impiega 6 clock (3 DIV da effettuare), dovremo calcolare il massimo fra
\[
3\cdot 4 = 12  ~~~~~ 4\cdot 2 = 8  ~~~~~ 6\cdot 3 = 18 
\]
A questo risultato vanno sommati 2 clock (uno per ID e uno per IF) e tanti cicli di WB quante sono le istruzioni effettuate (il WB � obbligatorio e solo un ciclo dopo la sua messa in atto una nuova istruzione precedentemente in \textit{ready} o in \textit{waiting} pu� partire! Questo implica un'attesa di un clock per ogni istruzione).
\item Il numero di cicli di clock impiegato nell'elaborazione nel caso puramente sequenziale (non in \textit{pipeline}) � invece pari a 
\[
3 + \sum_i \left( \text{Numero istruzioni di tipo \textit{i}} \cdot \text{Clock per istruzione di tipo \textit{i}} \right)
\]
dove il 3 contiene le fasi di ID, IF e WB.
Prendendo in considerazione l'esempio del punto precedente abbiamo quindi
\[
3 + (12 + 8 + 3) = 26~ \text{ clock}
\]
Un calcolo analogo pu� essere trovato nel paragrafo \ref{sec:viagra}.
\item Con $k$ CRB fino a $k$ istruzioni possono effettuare contemporaneamente un WB "'valido'".

\end{itemize}

\chapter{Quattro esempi di esercizi da 23 punti: commento alle soluzioni}
\label{cha:commentoSoluzioni}

Ove non espresso diversamente, i numeri [\ldots] si riferiscono al numero della slide della soluzione.

\section{Compito del 27 maggio 2004}
\label{sec:27mag04}

\subsection{\textit{Flow chart}}

Il programma deve:
\begin{itemize}
\item ricevere A in DMA;
\item eseguire A = A or B;
\item ricevere nuovamente A;
\item eseguire B = A and B.
\end{itemize}
Quando nei testi non si parla di PIC, e ci viene chiesto di andare a ricevere/trasmettere qualcosa tramite il DMAC, dobbiamo sempre fare riferimento alla soluzione \textit{a polling}: ci� significa che, per verificare se sia terminata la ricezione/trasmissione di un dato, sar� necessario interrogare il bit TC (\textit{terminal count}). Un po' differente � la soluzione imperniata sull'uso del meccanismo delle interruzioni: spesso, in tal caso, sar� necessario definire dei nuovi \textit{task} espressamente dedicati alla ricezione del messaggio. Nella soluzione a \textit{polling}, invece, spesso tutta la procedura da eseguire � "'condensata'" all'interno della \textit{flow chart} del \textit{main program}.
La \textit{flow chart} [5] viene quindi abbastanza naturale:
\begin{itemize}
\item prima dovremo inizializzare tutti i registri;
\item dopodich� si chiama l'\textit{initiator}: tramite esso sar� possibile ricevere A attraverso il DMAC;
\item a questo punto facciamo \textit{polling} sul bit TC. Se esso � pari a 0 vorr� dire che ancora il trasferimento non � completo e dovremo attendere; in caso contrario possiamo proseguire nel flusso d'istruzioni del programma; 
\item effettuiamo l'operazione OR;
\item nuova ricezione, nuova \textit{call} dell'\textit{initiator};
\item ancora una volta si fa \textit{polling} sul bit di TC;
\item una volta terminata anche la seconda transazione effettuiamo l'AND.
\end{itemize}
Si noti che l'\textit{initiator} va chiamato ogni volta che dobbiamo effettuare una transazione di I/O!

\subsection{Schema a blocchi}
\label{sec:schemaBlocchiMadre}

Lo schema a blocchi � abbastanza "'meccanico'" da disegnare: in tutti gli esercizi vi sono la CPU, la RAM e la EPROM connessi al bus dati (64 bit), nonch� il bridge che interfaccia tale bus con quello di I/O (8 bit). Quel che � collegato al bus di I/O dipende specificatamente dall'esercizio: pu� esserci un PIC, un DMAC, una o pi� periferiche (spesso porte seriali o porte parallele). In questo caso abbiamo semplicemente la porta P\_IN e il DMAC [8].

\subsection{Interfacciamento del DMAC}
\label{sec:DMACmadre}

L'interfacciamento del DMAC � un po' delicato soprattutto quando si utilizza una mappatura \textit{non contigua} dei dati (ovvero quando non tutti i banchi della memoria fisica vengono utilizzati). Purtroppo � questo il caso: A occupa infatti solo i banchi 0 e 1 (degli 8 totali), mentre B � mappato a indirizzi contigui.
Siccome B non verr� mai ricevuto tramite la porta seriale, possiamo modificare lo schema di interfacciamento del DMAC rendendolo specifico e completamente dedicato alla ricezione di A [13]. Il DMAC ha s� una rete in grado di generare i segnali di BE (\textit{bank enable}), tuttavia essa dev'essere alterata per far s� che indirizzi contigui per il DMAC (lui, \textit{pur�t}, sa solo indirizzare la memoria in modo incrementale o decrementale) siano in realt� indirizzi \textit{interleaved} verso la memoria (ovvero sul bus degli indirizzi), cosicch� a noi baster� programmare il BAR con l'indirizzo-base a cui si trova A e inserire nel BCR il numero di byte da trasmettere dalla porta seriale. Al resto, cio� a inter\textit{leave}are gli indirizzi, deve pensarci una rete logica che interfacci correttamente il DMAC al bus degli indirizzi. 
Non � per� finita qui: il DMAC deve poter essere programmato e/o gestito dalla CPU quando quest'ultima � \textit{master}. L'interfacciamento di cui sopra deve quindi essere \textit{bidirezionale}, nel senso che l'inter\textit{leave}aggio automatico degli indirizzi non deve essere operativo quando dalla CPU indirizziamo il DMA controller per leggere/scrivere nei suoi 16 byte "'interni'": per fare ci� esistono infatti i pin A[3\ldots 0], i quali devono poter essere interfacciati con i rispettivi pin BA [3\ldots 0] (del bus dati) \textit{senza alterazioni} (alterazioni che invece sar� necessario abilitare quando il DMAC � \textit{master}).
Noi siamo per� aspiranti ingegneri forti e scaltri\footnote{Beh, s�, insomma\ldots Ce la caviamo!} e non ci facciamo spaventare: possiamo infatti sfruttare a nostro vantaggio un segnale, HOLDA, che � in grado di dirci se comanda la CPU (HOLDA = 0) oppure il DMAC (HOLDA = 1). La presenza di tale segnale ci fa comodo perch� in questo modo la nostra rete logica inter\textit{leave}atrice potr� discriminare i due casi e funzionare bene in ogni situazione. \\

Passiamo ora ad esaminare la slide [13]. L'idea �: dobbiamo fare in modo che vengano continuamente selezionati i banchi 0 e 1. Byte per byte, gli indirizzi che il DMAC dovr� generare saranno quindi:
\begin{verbatim}
0000 8000H (banco 0)
0000 8001H (banco 1)
0000 0008H (banco 0)
0000 0009H (banco 1)
0000 8010H (banco 0)
0000 8011H (banco 1)
0000 0018H (banco 0)
0000 0019H (banco 1)
etc...
\end{verbatim}
Come si nota, dopo la scrittura di due byte in memoria dobbiamo "'saltare di 6'" per finire nella riga sopra (si pensi allo schema in [9] e [12]). Come ce la sgavagnamo? Proviamo ad immaginare il seguente interfacciamento, in cui A � un pin del DMAC e BA � un pin del bus indirizzi:
\begin{verbatim}
A0 --> BA0
A1 --> BA3 (!)
A2 --> BA4
A3 --> BA5
A4 --> BA6
e cos� via...
\end{verbatim}
Avremo la corrispondenza fra le seguenti coppie (indirizzo generato dal DMAC, indirizzo che va a finire sul bus degli indirizzi)\footnote{Solo per alcune cifre esadecimali, le meno significative in entrambi i casi!}:
\begin{verbatim}
DMAC              BUS INDIRIZZI
0... 00H          ... 00H
0... 01H          ... 01H
0... 02H          ... 08H (A1 finisce in A3 --> 2 = 0010 => 1000 = 8)
0... 03H          ... 09H (A1 finisce in A3 e A0 sta dov'� --> 3 = 0011 => 1001 = 9)
\end{verbatim}
Siamo quindi riusciti ad ottenere il risultato tanto agognato semplicemente collegando i pin dei due estremi (DMAC e bus) in maniera astuta!
Configurando quindi opportunamente il DECODER (che prende solo A0 per generare gli unici 2 BE possibili) e collegando i pin del DMAC successivi al primo ai relativi pin (ma \emph{sfasati di 2}!) del bus dati possiamo ottenere il comportamento desiderato quando il DMAC � \textit{master}. Anche la situazione in cui la CPU comanda il DMAC � salva, perch� ci sono l'ENCODER e un DRIVER a gestire rispettivamente gli otto segnali di BE (da tradurre in un numero binario di 3 bit, che costituiranno il BA[2\ldots 0]) e il segnale BA3 per l'indirizzamento dei 16 registri del DMAC.
In virt� della scelta fatta poco fa per interfacciare DMAC e BUS, al \textit{latch} HH (quello che genera gli 8 bit pi� significativi dei 32 che andranno a finire sul bus degli indirizzi) andranno collegati i soli pin IOB[5\ldots 0]. Il \textit{latch} HL, invece, riceve i soliti IOB[7\ldots 0]; sia HL e HH campionano in maniera asincrona ci� che ricevono dal bus di I/O una volta che LE (= CS\_HL\# nor IOWRC\# per HL e = CS\_HH\# nor IOWRC\# per HH), ovvero quando la CPU � \textit{master} e vuole programmare il DMAC. L'OE, invece, si attiva quando HOLDA e alto e quando quindi il DMAC � \textit{master} e vuole indirizzare la memoria da 32 bit.
Grazie ai \textit{latch} e all'uso degli 8 bit per i dati (D[7\ldots 0]) riusciamo a raggiungere i 32 necessari a indirizzare qualunque porzione della nostra memoria (o, meglio, questo � vero nel caso generale: questo DMAC modificato pu� unicamente raggiungere i banchi 0 e 1)!

\subsection{Programmazione del DMAC}

Anche questo punto � un po' delicato quando bisogna fare i conti con l'indirizzamento \textit{interleaved}. La tentazione � infatti quella di scrivere
\begin{verbatim}
BAR = 8000H
\end{verbatim}
in quanto nel testo si dice che A � ivi mappato. Ci� per� costituirebbe un errore in virt� di quanto detto nel paragrafo precedente: non vi � corrispondenza 1-a-1 fra i pin del DMAC e quelli del bus degli indirizzi! Dobbiamo quindi in qualche modo "'tradurre'" questo indirizzo in quello corretto che tenga conto della rete logica interposta fra DMAC e BUS. Considerando che si ha uno sfasamento di 2 posizioni fra i bit in uscita dal DMAC e quelli del BUS, l'indirizzo 8000H corrisponder� a
\begin{verbatim}
1000 0000 0000 0000 => 0010 0000 0000 0000
   8    0    0    0 =>    2    0    0    0
\end{verbatim}
Dovremo quindi scrivere
\begin{verbatim}
BAR = 2000H
\end{verbatim}
Il BCR, invece, non viene mai influenzato dalla questione dell'indirizzamento \textit{interleaved} e quindi basta inserire il numero di byte da trasferire - 1 (64 byte - 1 = 63 byte):
\begin{verbatim}
BCR = 40H - 1 = 3FH
\end{verbatim}


\subsection{Disegnare ai morsetti la porta P\_IN}

La periferica P\_IN � una porta parallela a 16 bit e quindi un dispositivo con $k = 2$: essa ha cio� 4 registri interni da 8 bit (la parte LSB del dato, la parte MSB del dato, il registro di stato e un quarto registro \textit{undefined}\footnote{Con un singolo bit non potevano indirizzare tre registri, mentre con due ne indirizziamo uno in pi� del necessario, il che � sovrabbondante, ma non possiamo farci niente!}) e sono necessari 2 bit per poterli indirizzare tutti quanti. In [10] possiamo vedere la tabella che indica le corrispondenze fra indirizzo e registro al quale si accede: risulta evidente che tale dispositivo occuper� 4 posizioni nello spazio di I/O. Ci� significa che, se abbiamo scelto 800H come indirizzo base al quale mappare la periferica, saranno a quest'ultima riservati gli indirizzi 800H\ldots 803H

Gli altri PIN importanti sono STROBE, utile alla sincronizzazione, DIN[15\ldots 0], per la ricezione dei dati dall'esterno, CS*, il \textit{chip select}, D[7\ldots 0], per la connessione al bus di I/O, IORD\#, da parte del bus dei comandi, e SRQ, per poter segnalare al DMAC che si vuole effettuare una transazione di I/O.


\subsection{Forme d'onda}
L'ordine di attivazione � il seguente (le forme d'onda sono riportate in [16] e si riferiscono ad A(30). Quelle di A(31) sono praticamente identiche, salvo pochissime modifiche):
\begin{itemize}
\item DRQ $\to$ 1: "'DMAC, devo trasmetterti qualcosa!'";
\item HRQ $\to$ 1: il DMAC accoglie la richiesta e fa si appella alla CPU per poter gestire il bus;
\item HOLDA $\to$ 1: la CPU, dopo aver finito quel che doveva finire, concede il permesso al DMAC;
\item arrivano gli indirizzi: i BE sono da intendersi attivi bassi quindi saranno tutti ad uno tranne l'ultimo (primo dato, FEH) e il penultimo (secondo dato, FDH). In BA ci andr� invece l'offset di A(30), che � 101EH
\begin{verbatim}
Indirizzo "effettivo":
8    0    F    0
1000 0000 1111 0000

Offset (l'indirizzo effettivo diviso per 8 --> 3 shift a DX):
1    0    1    E
-->1 0000 0001 1110 
\end{verbatim}
\item arrivano i comandi di lettura del primo byte dalla periferica (IORD\# si abbassa) e della sua scrittura in memoria (MEMWR\#);
\item arrivano gli stessi comandi, ma per il secondo byte: in seguito all'abbassamento di IORD\# si abbassa anche DRQ, perch� la periferica ha trasmesso tutto quello che aveva da trasmettere;
\item a questo punto si abbassano HRQ e DACK;
\item infine HOLDA torna basso e la CPU nuovamente � \textit{master}; si intravede infine la nuova risalita di DRQ per la ricezione degli ultimissimi 2 byte corrispondenti alla \textit{word} A(31).
\end{itemize}
Alla termine della ricezione dell'ultimo dato, l'EOP\# si attiver� ($\to$ 0) in corrispondenza dell'abbassamento di MEMWR\# dovuto alla ricezione dell'ultimo byte.

\subsection{\textit{Cache} e stato MESI}

Prima cosa da fare quando si tratta di scegliere dove andare a posizionare i nostri dati in \textit{cache}, trasferendoli dalla memoria fisica, � disegnarsi lo schemino con (TAG, SET\_ID, OFFSET): in questo modo potremo capire su quale linea di \textit{cache} (ve ne sono \textit{n} per SET\_ID, dove \textit{n} � il numero di vie) andare a posizionare i nostri dati.
Nel nostro caso:
\begin{verbatim}
**** PER A:
0      0      0      0      8      0      0      0    H 
0000   0000   0000   0000   1000   0000   0000   0000
|                              |   |        ||       |
|             TAG              |   | SetID  ||Offset |

TAG: 00008H   SetID: da 00H a 07H

**** PER B:
0      0      0      0      4      0      0      0    H 
0000   0000   0000   0000   0100   0000   0000   0000
|                              |   |        ||       |
|             TAG              |   | SetID  ||Offset |

TAG: 00004H   SetID: da 00H a 01H
\end{verbatim}

A, in virt� della sua stravagante posizione in memoria fisica, occuper� pi� linee di \textit{cache} di quanto si possa credere ad un primo acchito. Ogni linea, infatti, pu� ospitare fino a 32 byte, i quali vengono presi dalla memoria fisica tramite cicli \textit{burst}: ognuno di questi cicli porter� in \textit{cache} quattro blocchi di 8 byte (uno per banco), ma di questi 8 byte solo 2 conterranno il vettore A. Perci� A non occupa solo 2 linee di \textit{cache} (64 byte / 32 byte = 2) bens� 4 volte tante, cio� 8 (ecco perch� il setID va da 00H a 07H).
Il calcolo che era errato per A va invece bene per B in virt� della sua disposizione a indirizzi contigui (2 linee di \textit{cache}, setID da 00H a 01H).
Si noti inoltre che i due dati hanno in comune i setID 00 e 01: A e B non possono quindi stare sulla stessa via (vedi [25]).

L'evoluzione della \textit{cache} � la seguente:
\begin{itemize}
\item inizialmente � invalida (\textbf{I}) perch� non contiene il dato;
\item per effettuare l'operazione A = A or B la \textit{cache} ha bisogno di entrambi i vettori e, dopo una \textit{miss} obbligatoria, li carica nelle sue linee con stato MESI \textbf{E} (lo stato sarebbe invece \textbf{S} se lavorassimo con politica \textit{write true});
\item dovendo sovrascrivere A in \textit{cache} con il relativo valore aggiornato, pone il relativo stato MESI a \textbf{M};
\item riceviamo nuovamente A in DMA ma, nel fare ci�, la CPU \textit{snoop}pa e si accorge che si vuole sovrascrivere tale vettore in memoria centrale: prima che ci� avvenga, la \textit{cache} passa il dato alla RAM e setta lo stato MESI a \textbf{I} (invalido);
\item bisogna eseguire B = A and B, quindi bisogna ricaricare in \textit{cache} A. Lo stato MESI ripassa a \textbf{E};
\item effettuato il calcolo scriviamo in B, quindi A rimane in \textbf{E} e B passa in \textbf{M}.
\end{itemize}

\subsection{Cicli di bus esterno}

Per cicli di bus esterno si intendono quelli da/verso la memoria fisica.
Essi sono:
\begin{itemize}
\item quelli effettuati dal DMAC per scrivere A in memoria: A � un dato di 64 byte e il DMAC scrive un byte alla volta, per cui i cicli esterni sono $64\cdot 2 = 128$ (64 per ogni trasferimento);
\item quelli dovuti al \textit{Write Back Hit-Modified} (effettuati dalla CPU): essi sono 8, cio� uno per ogni linea di \textit{cache} di A. Infatti A � stato modificato (stato MESI = M) in seguito alla prima operazione vettoriale, cosicch� si � resa necessaria una sua copia in memoria fisica prima della sua seconda ricezione in DMA;
\item quelli dovuti alle \textit{miss obbligatorie} (effettuati dalla CPU), ovvero quelli necessari a trasferire A e B per la prima volta in memoria: in questo caso abbiamo 10 cicli \textit{burst} (8 per A e 2 per B).
\end{itemize}

\subsection{\textit{Miss-Rate} sui dati}

Quando si calcola la \textit{miss rate} bisogna sempre fare riferimento alla \textit{cache}: considerando di fare le nostre operazioni byte-per-byte abbiamo
\begin{itemize}
\item per fare l'OR - 32 accessi in lettura per leggere A, 32 accessi in lettura per leggere B, 32 accessi in scrittura per scrivere il vettore A risultante dall'operazione;
\item per fare l'AND - 32 accessi in lettura per leggere A, 32 accessi in lettura per leggere B, 32 accessi in scrittura per scrivere il vettore B risultante dall'operazione.
\end{itemize}
Il totale � quindi di 192 accessi. Si noti che non vengono considerati i cicli di \textit{write back hit-modified}.


Le \textit{miss} sono invece 18:
\begin{itemize}
\item una per ogni linea di \textit{cache} di A, per un totale di 8. Queste sono le \textit{miss} obbligatorie che servono per poter portare in memoria il vettore A: le miss sono 8 e non 64 perch�, in reazione ad ogni \textit{miss}, si ha un ciclo \textit{burst} che porta in \textit{cache} direttamente 8 byte del nostro vettore (1 miss = 8 byte trasferiti; 8 miss = 64 byte trasferiti!);
\item altre 8, sempre per A, visto che in seguito alla seconda ricezione la \textit{cache} � invalida e dobbiamo fare fronte ad ulteriori 8 cicli \textit{burst} per prelevare il dato aggiornato dalla RAM;
\item una per ogni linea di \textit{cache} di B, per un totale di 2.
\end{itemize}

Abbiamo quindi 18 miss e 192 accessi: la \textit{miss rate} sar� quindi pari a
\[
\dfrac{18}{192} = 9,4\% 
\]

\section{Compito del 21 giugno 2005}
\label{sec:21giu05}

\subsection{Schema a blocchi}
Quesito che non ha bisogno di molti commenti: si faccia riferimento al paragrafo \ref{sec:schemaBlocchiMadre} e alla slide [4].

\subsection{Schema di indirizzamento di I/O}
Altra domanda abbastanza meccanica e non molto impegnativa: essa consiste nel compilare una tabellina simile a quella in [5], in cui andare ordinatamente a elencare i componenti di I/O e le memorie. Si presti attenzione anche alla corretta definizione dei \textit{chip select} semplificati (vedere Calcolatori L-A).
\begin{itemize}
\item RAM: dimensione dipendente da quanto scritto nel testo del problema (quasi sempre 1 MB), segnali di controllo MEMWR\# e MEMRD\# (scriviamo e leggiamo);
\item EPROM: dimensione dipendente da quanto scritto nel testo del problema (quasi sempre 4 MB), segnali di controllo MEMRD\# (sola lettura!);
\item PIC: 2 byte in spazio di I/O, segnali di controllo IORDC\#, IOWRC\# e, non dimentichiamoci, INTA\#!
\item porta seriale: 8 byte in spazio di indirizzamento di I/O, segnali di comando IORDC\# e IOWRC\#;
\item DMAC: 16 byte in spazio di indirizzamento di I/O, segnali di comando IORDC\# e IOWRC\#;
\item una coppia di \textit{latch} per ogni canale del DMAC (in realt� ci� potrebbe rivelarsi ridondante nel caso i canali condividessero i bit in uscita da tali \textit{latch}): ogni \textit{latch} � mappato in 1 byte di spazio di indirizzamento di I/O ed � gestito dal segnale di comando IOWRC\# (la CPU pu� solo programmarli, � il DMAC che li legge quando HOLDA � alto).

\end{itemize}

\subsection{Interfacciamento del DMAC}

Dobbiamo memorizzare i nostri messaggi agli indirizzi $i0000$H per cui il \textit{latch} per gli 8 MSB (corrispondenti alle 2 cifre esadecimali pi� significative dell'indirizzo) non ha molte responsabilit� visto che essi sono sempre pari a 00H durante tutta la transazione. L'interfacciamento, mostrato in [6], � quindi abbastanza banale.

Non � invece cos� banale l'interfacciamento del \textit{latch} che gestisce i bit di indirizzo BA[23\ldots 16]. Sua � infatti la gestione della 3� e della 4� cifra esadecimale dell'indirizzo (contando a partire dalle pi� significative): quest'ultima, in particolare, assumer� valori da 1 a F per un totale di 16 configurazioni diverse (una per messaggio).
L'interfacciamento � mostrato in [7]. 
\begin{itemize}
\item Quello che dovrebbe essere il nostro \textit{latch} viene in realt� implementato tramite un contatore. Questo non toglie che sar� necessario settare un valore di partenza del conteggio, alla stregua come viene campionato il valore del \textit{latch} all'attivazione del segnale LD: in effetti la figura mostra che, quando sono attivi il segnale IOWRC\# e il \textit{chip select} del nostro contatore, il contatore si inizializza con ci� che arriva dall'I/O bus. 
\item Questo contatore deve poter essere incrementato ad ogni ricezione. L'incremento avviene in maniera automatica ogni volta che si ha un fronte (in questo caso negativo) del clock: il segnale che funge da clock � l'OR fra il segnale di MEMWR\# e il DACK\# del canale 0 (ci� significa che viene incrementato l'indirizzo quando il DMAC sta diventando \textit{master} e vuole scrivere in memoria ci� che viene gestito dal canale 0; questo avviene quando il DMAC sta iniziando a trasferire un messaggio, quindi la scelta � corretta).
\item Dobbiamo essere in grado di inviare un segnale al PIC per segnalare l'avvenuta ricezione di ognuno dei nostri messaggi. Per fare ci� possiamo sfruttare il carattere 0DH inviato dalla periferica, il quale � preposto a segnalare proprio tale eventualit�; esso viene quindi dato in pasto a un comparatore in grado di confrontare tale valore con ci� che passa sull'I/O bus. Se sta arrivando il carattere di fine transazione, quindi, il comparatore restituisce 1: tale segnale viene messo in NAND con MEMWR\footnote{Il prof. ha confermato che c'� un errore nella slide.} (attenzione: MEMWR � 1 quando si scrive in memoria!) e con DACK (anche questo segnale, si noti, � preso ora in logica positiva) cosicch� viene segnalato l'EOP\# quando DACK, MEMWR e il risultato del comparatore valgono contemporaneamente 1, il che significa che stiamo scrivendo in memoria (il DMAC � ancora master), che stiamo trasferendo proprio da quella periferica (usiamo il DACK del canale 0, cio� verso la nostra periferica) e che sta finendo il messaggio (a ci� serve il carattere 0DH).
\item L'uscita del comparatore � fornisce anche l'\textit{enable} che permette al contatore di incrementarsi (in fin dei conti dobbiamo aggiornare l'indirizzo, cio� il valore del contatore, alla fine di ogni messaggio per prepararlo alla ricezione del successivo). Inoltre, tale segnale viene dato in pasto alla piccola rete logica della parte SX di [7], la quale genera un interrupt (setta IR0 del PIC) al termine della ricezione del quindicesimo messaggio. Se il contatore vale infatti 0FH = 0000 1111, i quattro "'uni'" assieme al segnale del comparatore fanno campionare un '1' al Flip-flop D e permettono l'attivazione di IR0 quand'� il momento (cio� quando sono stati ricevuti tutti e 15 i messaggi).

\end{itemize}

Per il resto dell'interfacciamento e ulteriori dettagli si vedano le slide del prof e il paragrafo \ref{sec:DMACmadre}: in questo caso, tra l'altro, andiamo a memorizzare il dato a indirizzi contigui quindi non c'� bisogno di effettuare un interlacciamento "'speciale'" fra DMAC e bus degli indirizzi (il che rende il caso ancora pi� semplice di quello mostrato nel paragrafo \ref{sec:DMACmadre}).

\subsection{Programmazione del DMAC}

Programmare il DMAC significa definire il contenuto di BAR, BCR e la configurazione del \textit{register mode}.

Nel BAR vanno scritti i 16 bit meno significativi dell'indirizzo base al quale andremo a porre le nostre informazioni in memoria fisica (agli altri 16 bit pensano i \textit{latch}).
Nel nostro caso, dovendo andare a porre i messaggi a $i0000$H, lasceremo il BAR a 0000H.
Nel BCR dobbiamo invece andare a inserire il numero di byte da trasferire (meno 1) per ogni messaggio e quindi
\begin{verbatim}
BCR = 1K - 1 = 3FFH
\end{verbatim}

Imposteremo poi il \textit{register mode} con: 
\begin{itemize}
\item \textit{single mode};
\item indirizzi autoincrementanti\footnote{Orribile, detto in italiano.};
\item \textit{write-to-mem};
\item modalit� \textit{autoinit} attivata;
\item canale 0.
\end{itemize}

\subsection{Integrazione della porta seriale}

L'aspetto fondamentale dell'integrazione delle 2 porte seriali 8250 risiede nell'inserimento di 3 MUX, uno per A0, uno per A1 e uno per A2 (necessari per pilotare i segnali che andranno in ingresso alla seriale) e nella definizione del CS. I tre \textit{multiplexer} saranno comandati da HOLDA: se HOLDA = 1 significa che � il DMAC ad avere il campo quindi A0, A1 e A2 vengono posti a massa. In caso contrario, la CPU deve poter indirizzare i registri interni della porta seriale 8250 e quindi A0, A1 e A2 vengono collegati ai rispettivi pin del bus degli indirizzi.

Abbiamo poi bisogno di un altro \textit{multiplexer} per comandare il \textit{chip select} della nostra porta seriale: nel caso infatti che sia \textit{master} la CPU, il CS � quello indicato nella tabella di indirizzamento dei dispositivi (/BA10 /BA9 BA8 o, in logica legata, visto che in figura [9] CS � attivo basso, BA10 + BA9 + /BA8\footnote{Sono state utilizzate le leggi di De Morgan.}); se invece sta lavorando il DMAC, il CS � dato dal DACK.

\subsection{Mostrare il contenuto di PT, PD e CR3}

Forse questo � uno dei punti pi� "'delicati'" dell'intero esercizio, non tanto per la difficolt� ma per l'attenzione che richiede nel calcolo degli indirizzi.


Si faccia riferimento a [10]: nel testo si dice che negli ultimi 8 KB di memoria fisica sono presenti la PD (\textit{Page Directory}) e la PT (\textit{Page Table}) dei primi 4 MB di memoria virtuale. Sia la PT che la PD han dimensione pari a 4 KB quindi possiamo scegliere di porre la PD all'indirizzo 000F E000H e la PT a 000F F000H. I quindici messaggi da memorizzare si trovano per ipotesi agli indirizzi $i0000$H (della memoria fisica, $i$ = indice del messaggio) quindi riescono a stare tutti nei primi 4 MB di memoria virtuale (che terminano all'indirizzo 3F FFFFH). Questo significa che la PT mappata a FF000H sar� grande a sufficienza da contenere tutte le quindici \textit{entry} che si riferiscono ai nostri messaggi.

In CR3 vanno inserite le 5 \textit{most significant} cifre esadecimali dell'indirizzo della \textit{page directory} (gli altri 12 bit sono automaticamente a 0 perch� la PD � grande 4 KB ed � posizionata ad indirizzi allineati).
Considerando:
\begin{itemize}
\item che ogni PDE contiene il riferimento ad una \textit{Page Table} (anche lei di 4 KB), la quale conterr� 1 K \textit{entry} (dette PTE) che a loro volta si riferiranno alle pagine vere e proprie (pure loro di 4 KB),
\item che i nostri messaggi si trovano tutti nei primi 4 MB di memoria virtuale, 
\end{itemize}
l'entry da inserire in PD � dovr� puntare alla PT che abbiamo diligentemente mappato all'indirizzo fisico FF000H (e che � la prima, come dicevamo poco sopra) in quanto l� si troveranno le PTE puntanti alle pagine di 4 KB coi messaggi veri e propri. 

Siccome ogni \textit{entry} della \textit{page table} � di 4 byte, come facciamo a sapere quali sono le righe della PT che contengono i riferimenti ai nostri messaggi?
Anzitutto cerchiamo di capire l'ID delle pagine contenenti i dati tratti da DMAC: se ogni pagina � di 4 KB, per trovare l'ID della pagina (cio� il numero della pagina, supponendo di numerarle in maniera incrementale a partire da quella a indirizzi pi� bassi) basta semplicemente "'segare'" le tre cifre esadecimali meno significative dell'indirizzo in memoria virtuale:
\begin{verbatim}
PAGINA "numero 0" : indirizzo IV iniziale --> 0000 0000 H --> Page_ID = 0000 0 H
PAGINA "numero 1" : indirizzo IV iniziale --> 0000 1000 H --> Page_ID = 0000 1 H
...

MESSAGGIO 1:  0001 0000 H --> Page_ID = 0001 0H
MESSAGGIO 2:  0002 0000 H --> Page_ID = 0002 0H
...
MESSAGGIO 15: 000F 0000 H --> Page_ID = 000F 0H
\end{verbatim}
Il Page\_ID � anche pari all'ID della PTE, cio� al numero della "'riga'" della \textit{Page Table}. Il primo messaggio sar� quindi alla 10H-sima riga (cio� alla 16�, contando da 0) e verr� riferito dalla 10H-sima \textit{Page Table Entry}, il secondo alla 20H-sima (cio� alla 32�) e verr� riferito dalla 20H-sima \textit{Page Table Entry}, e cos� via. 

Ora si tenga presente che ogni PTE occupa 4 byte; detto questo, � semplice capire a quale indirizzo fisico si trovino le \textit{entry}: basta infatti moltiplicare per 4 il Page\_ID per calcolare l'offset all'interno della PT. E infatti:
\begin{verbatim}
4 byte ogni PTE
                                                      BASE      OFFSET      IF
MESSAGGIO 1:  Page_ID = PTE_ID = 0001 0H @ indirizzo FF000H + (10H x 4) = FF040H
MESSAGGIO 2:  Page_ID = PTE_ID = 0002 0H @ indirizzo FF000H + (20H x 4) = FF080H
...
MESSAGGIO 15: Page_ID = PTE_ID = 000F 0H @ indirizzo FF000H + (F0H x 4) = FF3C0H
\end{verbatim}

In tutte le \textit{entry} specificate (sia quella in PD sia quella nelle PT) andremo a mettere il bit \textit{Present} a 1.

\subsection{Valori dei TAG e dei SET\_ID}

Punto abbastanza banale, oserei dire. Basta prendere gli indirizzi virtuali dei primi due messaggi:

\begin{verbatim}
Primo messaggio:   IV da 10000H a 103FFH
Secondo messaggio: IV da 20000H a 203FFH


BIT                  31         ...            12 11   ...     5  4  ...      0
                     |          TAG             |  |  SET ID   |  |  OFFSET   |
1� messaggio: 10000H   0000 0000 0000 0001 0000     00H --> 1FH       ---         
2� messaggio: 20000H   0000 0000 0000 0010 0000     00H --> 1FH       ---
\end{verbatim}
Il set\_ID iniziale e finale � stato calcolato tenendo conto che i messaggi sono di 1 KB e quindi iniziano e terminano agli indirizzi riportati nello schemino soprastante.

\subsection{Si stimi il numero di TLB \textit{miss} e il numero di \textit{miss} nella \textit{cache} dei dati. Stato MESI}

Iniziamo con le \textit{miss} in \textit{Translation Look-aside Buffer}: esse sono obbligatorie in quanto l'elaborazione richiesta (l'OR byte-per-byte dei primi due messaggi) richiede operandi che si trovano in due pagine di memoria virtuale; alla fine dell'operazione il TLB conterr� le corrispondenze tra IV e IF della pagina del primo messaggio (che � la stessa in cui si trova V, quindi prendiamo due piccioni con una fava) e della pagina del secondo messaggio: questi due elementi sono stati portati in TLB in seguito a \textbf{due} miss. \\

Per quanto riguarda la \textit{miss-rate}:
\begin{itemize}
\item numero di accessi: 1024 per A, 1024 per B (dobbiamo leggere ogni byte dei due operandi), 1024 per V (scriviamo un byte alla volta del risultato);
\item numero di \textit{miss}: 1024/32 = 32 per prelevare ciascun messaggio (attenzione: una \textit{miss} in lettura provoca un ciclo \textit{burst} in grado di trasferire 32 byte, cosicch� si ha una \textit{miss} ogni 32 byte da trasferire), per un totale di 64 in lettura + 1024 \textit{miss} in scrittura (una per ogni \textit{byte} di V, come capita sempre nel \textit{write-around} quando si devono fare operazioni byte-per-byte) dovute al fatto che V non � presente in memoria e la politica di \textit{write-around} costringe il processore a "'scavalcare'" la \textit{cache} per scrivere il risultato direttamente in memoria fisica.
\end{itemize}

Si ha quindi una \textit{miss-rate} pari a 
\[
\dfrac{1024+64}{1024\cdot 3} = 35\%
\]

Siccome A e B vengono portati in memoria e mai alterati (scriviamo infatti su V), lo stato mesi sar� pari ad E per tutti.

\section{Compito del 20 marzo 2008}
\label{sec:20mar08}

\subsection{Domande preliminari}

\textsf{Nell'ipotesi che la velocit� delle porte seriali sia 500 Kb/s, quante matrici al secondo
vengono ricevute da S?}

Ogni matrice contiene 4 x 4 word (per un totale di 32 byte). 
Le matrici al secondo ricevute da S sono quindi:
\[
\dfrac{500\cdot 1024}{32\cdot 8}=2000
\]

\textsf{Nell'ipotesi che le CPU vadano a 100 MHz, e che ogni moltiplicazione richieda 20
periodi di clock, quale percentuale del tempo disponibile � trascorso dalle CPU nel
calcolo dei prodotti matriciali?}

Un periodo di clock � lungo:
\[
\tau = 10^{-8} ~\text{s}
\]
Quindi ogni moltiplicazione (elementare) impiega 200 nanosecondi per essere eseguita. Il prodotto fra due matrici 4x4 richiede tuttavia 64 moltiplicazioni elementari (ogni elemento della prima matrice viene moltiplicato per quattro elementi della seconda, quindi le moltiplicazioni sono $16 \cdot 4 \cdot 4 = 64$), dunque il tempo richiesto per una moltiplicazione matriciale �:
\[
200 \cdot 64 = 12800 ~\text{ns} = 12,8 ~\text{us} 
\]
Il tempo di ricezione di una matrice � tuttavia 
\[
\dfrac{32 \cdot 8}{500\cdot 1024}= 500 ~\text{us} 
\]
Quindi il tempo speso a effettuare le operazioni di moltiplicazione � pari al 
\[
100 \cdot \dfrac{12,8}{500}= 2,5 \%
\]
del tempo totale. Volendo, possiamo approssimare al 3\% se consideriamo che, nel calcolo del prodotto matriciale, dobbiamo effettuare anche delle addizioni (le quali, tuttavia, sono molto pi� rapide delle moltiplicazioni) e vi saranno delle \textit{miss}.

Un'architettura del genere � fortemente sbilanciata: non serve avere un processore cos� potente se poi la ricezione delle matrici � cos� lenta. Si potrebbe ridurre in modo molto pi� che lineare il consumo del sistema riducendo la frequenza di funzionamento, ad esempio fino a 10MHz, e riducendo di conseguenza anche la tensione di alimentazione delle CPU.

\subsection{\textit{Data segment}}
\label{sec:dataSegmentesPlay}

Tutte le quantit� che faranno parte del DS sono matrici rappresentate come vettori di 16 word: la sua definizione, dunque, � abbastanza immediata:
\begin{verbatim}
A      dw  16dup(?)
B      dw  16dup(?)
M_IN   dw  16dup(?)
T1     dw  16dup(?)
SCENE  dw  16dup(?)
T2     dw  16dup(?)
VIEW   dw  16dup(?)
M_OUT  dw  16dup(?)
***********
dw = Double Word
16dup(?) = 16 DWord inizializzate a 0
\end{verbatim}

Questo segmento di dati avr� un descrittore (mostrato in [16]), semplicissimo da compilare in quanto � sufficiente andare a sbirciare sulle slide del prof per trovarne la struttura e la descrizione di tutti gli attributi.
L'unico (relativo) sforzo sta nel definire la base e il limite di tale segmento:
\begin{itemize}
\item BASE: ce la dice il testo, � 4000H;
\item LIMITE: � il numero di byte del segmento (meno uno). Abbiamo 8 elementi, ognuno contenente 16 elementi da 2 byte ciascuno, dunque il limite � pari a  $8\cdot 16 \cdot 2 = 256$ byte meno uno. Scriveremo dunque nel relativo campo FFH (= 255).
\end{itemize}

\subsection{\textit{Flow Chart}}

Punto un po' delicato per il numero relativamente elevato di cose di cui tener conto.
Anzitutto dovremo definire due \textit{flow chart}, una per ogni processore, visto che le operazioni eseguite dai due Pentium sono diverse.
Il primo processore:
\begin{itemize}
\item riceve tramite S1 le matrici e le memorizza in M\_IN,
\item copia M\_IN in T1,
\item esegue il prodotto matriciale ponendo il risultato in SCENE (e nel frattempo pu� gi� ricevere un'altra matrice),
\item commuta FF1\_2.
\end{itemize}
Il secondo processore:
\begin{itemize}
\item interroga a \textit{polling} il FF1\_2,
\item copia SCENE in un vettore M e commuta FF2\_1 per segnalare che SCENE � libero,
\item esegue il prodotto matriciale e lo pone in VIEW,
\item copia VIEW in M\_OUT, 
\item invoca un trasferimento in DMAC.
\end{itemize}

\begin{verbatim}
Flow Chart P1:

1 - Initialization. ** Questa ci va sempre!
2 - Programmazione del DMAC e call dell'initiator. 
     ** Ad ogni transazione bisogna chiamare l'initiator
3 - TC = 1? Se no, ricontrolla. Se s�, prosegui.
4 - Copia di quanto ricevuto dal DMAC in T1.
5 - Call initiator del DMAC. 
     ** Intanto che moltiplichiamo la periferica pu� trasferire un altro dato
6 - Esecuzione del prodotto matriciale.
7 - Commutazione del FF1_2. 
     ** Per segnalare che SCENE � pronto
8 - FF2_1 � pari a zero? Se no, ricontrolla. Se s�, torna all'istruzione 3. 
     ** Prima di poter ripartire dobbiamo essere sicuri che SCENE sia libero

Flow Chart P2:
1 - Initialization.
2 - FF1_2 � a 1? Se no, ricontrolla. Se s�, prosegui.
3 - Copia SCENE in T2.
4 - Commuta FF2_1.
5 - Esegue il prodotto matriciale e lo pone in VIEW.
6 - Copia VIEW in M_OUT.
7 - Chiama l'initiator.
8 - TC = 1? Se no, ricontrolla. Se s�, ricomincia dall'istruzione 2.
\end{verbatim}

Volendo, si pu� fare una piccola modifica alla \textit{flow chart} di P2: siccome la trasmissione in DMAC � un collo di bottiglia conviene, a partire dalla seconda ricezione in poi, modificare la \textit{flow chart} di P2 come segue:

\begin{verbatim}
Flow Chart P2:
1 - Initialization.
2 - FF1_2 � a 1? Se no, ricontrolla. Se s�, prosegui.
3 - Copia SCENE in T2.
4 - Commuta FF2_1.
5 - Esegue il prodotto matriciale e lo pone in VIEW.
6 - TC = 1? Se no, ricontrolla. Se s�, ricomincia dall'istruzione 2.
7 - Copia VIEW in M_OUT.
8 - Chiama l'initiator.
\end{verbatim}
In questo modo il DMAC pu� trasferire il dato intanto che vengono effettuate altre operazioni come la copia di SCENE in M e il prodotto matriciale.

\subsection{I \textit{flip-flop}: interfacciamento e \textit{routine} di commutazione}

L'interfacciamento � mostrato in [24]: in realt� esso, a mio parere, � fin troppo esaustivo. Sono presenti:
\begin{itemize}
\item i due \textit{flip-flop};
\item i componenti necessari sfruttare i segnali di comando per l'arbitraggio dei \textit{flip-flop};
\item una rete logica programmabile PAL per la generazione dei \textit{chip select}.
\end{itemize}
Probabilmente era sufficiente mostrare i flip-flop e la rete generatrice dei segnali di \textit{Clear} e di \textit{Clock}. Si noti che, per come sono stati interfacciati, i FF commutano (Q negato retroazionato in D) all'arrivo del fronte di clock; grazie ai buffer 3-state, inoltre, non pu� esservi conflitto sull'IO bus.

La \textit{routine} di commutazione � molto banale ed � mostrata in [26].

\subsection{\textit{Cache} e \textit{miss}}

Eccola, la vera bestia nera di questo compito. Nel rispondere a questa domanda bisogna in particolare tenere presente che:
\begin{itemize}
\item dobbiamo tenere presente la soluzione riportata nel paragrafo  \ref{sec:dataSegmentesPlay} per capire dove veramente si trovano i dati;
\item abbiamo due CPU e quindi \textbf{dobbiamo schematizzare due diverse \textit{cache}}!
\end{itemize}

Anzitutto ci vengono chiesti \textit{tag} e \textit{set\_ID} delle linee di \textit{cache} associate alle matrici definite nel DS. Questa parte non dovrebbe creare difficolt�:

\begin{verbatim}

DS: da 4000H a 40FFH

0      0      0      0      4      0      0      0    H 
0000   0000   0000   0000   0100   0000   0000   0000
0      0      0      0      4      0      F      F    H 
0000   0000   0000   0000   0100   0000   1111   1111
|                              |   |        ||       |
|             TAG              |   | SetID  ||Offset |
|                              |   |00 -> 07|

TAG: 00004H   SetID: da 00H a 07H
\end{verbatim}

Abbiamo quindi 8 possibili linee di \textit{cache}, ognuna delle quali conterr� interamente uno dei dati dichiarati nel \textit{Data Segment} (ogni matrice occupa esattamente 32 byte). \\
Il processore 1 avr� in \textit{cache}: M\_IN, T1, SCENE, A. \\
Il processore 2 avr� in \textit{cache}: M\_OUT, T2, SCENE, VIEW, B. \\
La disposizione � mostrata in [28] e richiede un attenta analisi delle posizioni dei dati dichiarati in  \ref{sec:dataSegmentesPlay}.

La dinamica dello stato MESI � un po' pi� complicata:
\begin{itemize}
\item il DMAC riceve le informazioni in M\_IN. La CPU nel frattempo sta \textit{snoop}pando e quindi invalida il dato in \textit{cache} se � presente (durante la prima transazione il dato non � presente quindi l'invalidazione avviene dalla seconda ricezione in poi);
\item la CPU ha bisogno di portare M\_IN in \textit{cache} per copiarlo in T1: lo stato MESI, per le linee associate a M\_IN, diventa E;
\item alla prima transazione, le linee di \textit{cache} che si riferiscono a T1 non saranno presenti in memoria (stato I). Dal momento che viene effettuata la copia M\_IN $\to$ T1, quelle linee passeranno quindi allo stato E. Dalla seconda ricezione in poi, invece, le linee saranno inizialmente gi� presenti quindi passeranno in M una volta avvenuto il processo di copia;
\item SCENE � inizialmente non presente in memoria (stato I, per entrambe le CPU): la cosa curiosa, tuttavia, � che in \textit{cache} non ci andr� mai a causa della politica di \textit{write-around}. SCENE, infatti, non essendo presente, viene direttamente scritto in memoria fisica! Dopo la moltiplicazione, il dato sar� I in P2 (nella sua \textit{cache} SCENE � presente, a differenza di quanto avviene per P1), in quanto la seconda CPU - dedita allo \textit{snooping} - si sar� accorta che qualcuno, altrove, ha scritto il dato aggiornato. Una volta ricevuto il dato aggiornato, lo stato MESI passer� ad E;
\item SCENE viene copiato in T2: quel che avviene � esattamente identico a quanto illustrato nel terzo punto (copia di M\_IN in T1);
\item idem per il calcolo di VIEW: la prima volta non c'� (I), poi viene portato in memoria (E) e infine modificato ad ogni calcolo (M);
\item M\_OUT, non essendo in memoria, viene scritto in \textit{write-around}. Come SCENE, non capiter� mai che finisca in \textit{cache}.
\end{itemize}

In base a quanto detto fin'ora:
\begin{itemize}
\item non ci sono cicli di \textit{Hit-Modified WB} perch� gli unici vettori condivisi tra pi� master (M\_IN, SCENE ed M\_OUT) non si trovano mai nello stato M: M\_IN viene solamente letto da una CPU (P1), SCENE viene scritto da P1, ma in \textit{Write Around} cos� come M\_OUT viene scritto in \textit{Write Around} da P2;
\item non ci sono cicli di \textit{Cache-Replacement WB} in quanto nella situazione considerata non ci sono situazioni di conflitto sulla \textit{cache}.
\end{itemize}

Calcoliamo ora il \textbf{numero di accessi per P1}, nel calcolo delle prime 5 moltiplicazioni matriciali ("'unit� di misura'": word):
\begin{itemize}
\item $16\cdot 5$ (lettura) per leggere in \textit{cache} M\_IN;
\item $16\cdot 5$ (scrittura) per scrivere in \textit{cache} T1;
\item $16 \cdot 2 \cdot 5$ (lettura) per eseguire la moltiplicazione (leggiamo entrambi gli operandi);
\item $16\cdot 5$ per "'tentare'" di scrivere SCENE (in realt� si genera una \textit{miss} in scrittura).
\end{itemize}
TOTALE: 240 accessi in lettura, 160 accessi in scrittura, 400 accessi in totale. \\
\textbf{Numero di accessi per P2:}
\begin{itemize}
\item $16\cdot 5$ (lettura) per leggere SCENE;
\item $16\cdot 5$ (scrittura) per scrivere SCENE in T2;
\item $16 \cdot 2 \cdot 5$ (lettura) per eseguire la moltiplicazione (leggiamo entrambi gli operandi);
\item $16\cdot 5$ (scrittura) per scrivere VIEW;
\item $16\cdot 5$ (lettura) per leggere VIEW (si vuole copiarlo in M\_OUT);
\item $16\cdot 5$ per per "'tentare'" di scrivere M\_OUT (in realt� si genera una \textit{miss} in scrittura).
\end{itemize}
TOTALE: 320 accessi in lettura, 240 accessi in scrittura, 560 accessi in totale.\\
\textbf{Passiamo alle \textit{miss} per P1:}
\begin{itemize}
\item $1\cdot 5$ (lettura) per portare in memoria M\_IN;
\item 16 (scrittura) per \textit{write-around} di T1 (solo prima volta, poi con la \textit{miss} in lettura lo portiamo dentro);
\item 1 (lettura, \textit{compulsory}) per portare in memoria A;
\item 1 (lettura, \textit{compulsory}) per portare in memoria T1;
\item $16\cdot 5$ (scrittura) per \textit{write-around} di SCENE.
\end{itemize}
TOTALE: 7 \textit{miss} in lettura, 96 \textit{miss} in scrittura, 103 \textit{miss} in totale. \\
\textbf{Infine elenchiamo le \textit{miss} per P2:}
\begin{itemize}
\item $1\cdot 5$ (lettura) per portare in memoria SCENE;
\item 16 (scrittura) per \textit{write-around} di T2 (solo prima volta, poi con la \textit{miss} in lettura lo portiamo dentro);
\item 1 (lettura, \textit{compulsory}) per portare in memoria T2;
\item 1 (lettura, \textit{compulsory}) per portare in memoria B;
\item 16 (scrittura) per \textit{write-around} di VIEW, scritto in seguito alla moltiplicazione (solo prima volta, poi con la \textit{miss} in lettura lo portiamo in \textit{cache});
\item 1 (lettura, \textit{compulsory}) per portare in memoria VIEW;
\item $16\cdot 5$ (scrittura) per \textit{write-around} di M\_OUT.
\end{itemize}
TOTALE: 8 \textit{miss} in lettura, 112 \textit{miss} in scrittura, 120 \textit{miss} in totale. \\

La \textit{miss-rate} complessiva � dunque:
\[
\dfrac{103+120}{400+560}=23,2\%
\]

\subsection{\textit{Interrupt}}
Si veda [32], in cui la risposta viene spiegata esaustivamente.

\section{Compito del 23 marzo 2006}
\label{sec:23mar06}

Forti dell'esperienza maturata nei paragrafi \ref{sec:27mag04}, \ref{sec:21giu05} e \ref{sec:20mar08}, possiamo di quest'ultimo compito d'esame commentare solo le parti veramente diverse rispetto a quelli precedenti.
Nulla di nuovo vi � infatti fino al quesito sui \textit{task}. Dopodich� le domande 5, 6 e 7 sono molto ben spiegate in [17-21], quindi in questa sede ne verr� tralasciata la trattazione.

\subsection{\textit{Flow-chart} dei \textit{task}}
Abbiamo cinque \textit{task} da definire: il primo dal quale partire � sicuramente INIT\_POI\_IDLE il quale:
\begin{verbatim}
1. Inizializza i flip-flop
2. Chiama TS_1
3. Chiama TS_2
4. HALT (basso consumo energetico)
\end{verbatim}

TS1 e TS2 sono praticamente identici:
\begin{verbatim}
1. Call initiator (DMA) nel rispettivo canale
2. IRET (intanto che il DMAC trasferisce perch� non possiamo fare altro?)
** ARRIVA L'EOP! Il DMAC ha finito la transazione. **
3. Settiamo il flip-flop
4. EOI, IRET
\end{verbatim}

Passiamo al task \textit{timer};
\begin{verbatim}
1. Controlliamo i flip-flop: sono entrambi attivi? 
	1b. Se NO, EOI e IRET: il task deve rilasciare il controllo.
2. Se s�, resettiamo i flip-flop e chiamiamo E.
3. EOI, IRET
\end{verbatim}

Infine rimane E: questo \textit{task} deve effettuare la somma e poi richiamare i due \textit{task} per la ricezione (TS1 e TS2).
\begin{verbatim}
1. Somma M1 e M2
2. Chiama TS1
3. Chiama TS2
4. EOI, IRET
\end{verbatim}

Vi sono due casi in cui sono presenti tre \textit{task busy}; per trovarli pu� essere interessante utilizzare il seguente metodo: si prova a simulare l'andamento dei processi nella CPU e si cerca di vedere quanti di loro riescono a percorrere contemporaneamente la \textit{flow-chart} nella parte precedente l'IRET.
\begin{itemize}
\item Nel primo caso ci sono INIT\_POI\_IDLE, TS\_1 e poi E (oppure INIT\_POI\_IDLE, TS\_2 e poi E, ma mai TS\_1 e TS\_2 contemporaneamente perch� � strutturalmente impossibile). 
Infatti INIT\_POI\_IDLE � sempre attivo, poi potrebbe essere \textit{busy} E in virt� del fatto che un \textit{interrupt} del clock ha fatto partire la somma e infine E, prima di segnalare l'IRET, potrebbe aver chiamato TS\_1 per far ripartire il processo di ricezione;
\item nel secondo caso abbiamo una situazione simile alla precedente, ma il verificarsi di un \textit{interrupt} del \textit{timer} ci porta ad avere 4 \textit{task} contemporaneamente \textit{busy}. TS\_1 (o TS\_2) diventa \textit{ready} per far posto un attimo al \textit{timer}, il quale uscir� senza attivare E perch� non � possibile che i due task TS\_1 e TS\_2 abbiano gi� settato il loro FF (altrimenti avremmo un \textit{fault} per rientro in E).
\end{itemize}

\subsection{Contenuto di IDT}

Nella IDT vi sono tre \textit{entry} in virt� del fatto che sono possibili tre tipi di \textit{interrupt}:
\begin{itemize}
\item l'\textit{interrupt} del \textit{timer};
\item quello di fine ricezione del messaggio per S\_1;
\item quello di fine ricezione del messaggio per S\_2.
\end{itemize}

Per definire le\textit{ entry} basta andare sulle \textit{slide} a copiare la forma del descrittore di \textit{task gate} (vedi [12]): volendo, si pu� definire ogni \textit{entry} utilizzando l'assembler: praticamente quasi tutti i campi sono messi a 0, tranne gli 8 bit che nel descrittore stanno in posizione centrale. Siccome il livello di privilegio � pari a 0 per tutti, essi assumeranno il valore 85H (1000 0101).

Nella parte inferiore della slide [12] sono elencati i tre descrittori (pi� il "'descrittore 0'") impostati col giusto ID (2, 3 e 4: li si tenga presente perch� dopo quest'ordine andr� rispettato nella GDT).

\subsection{Contenuto di GDT}

Contiene i descrittore dei \textit{Task State Segment} per i cinque nostri possibili \textit{task}; inoltre, contiene i riferimenti alle LDT di questi ultimi (le quali conterranno i descrittori dei segmenti di codice, dati e \textit{stack} associati a ciascun \textit{task}).
In [16] sono mostrati gli oggetti presenti: oltre al solito "'descrittore 0'" vi sono \textbf{rigorosamente nell'ordine} (vista la definizione in IDT) TSS\_E (ID = 1), TSS\_TIMER (ID = 2, come preventivato), TSS\_TS\_1 (ID = 3, idem), TSS\_TS\_2 (ID = 4, idem), etc\ldots


\listoffigures

\end{document}
